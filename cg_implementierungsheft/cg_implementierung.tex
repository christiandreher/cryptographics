\documentclass{article}

\usepackage[utf8]{inputenc}
\usepackage[ngerman]{babel}
\usepackage[ngerman]{translator}
\usepackage[T1]{fontenc}
\usepackage{enumitem}
\usepackage{graphicx}
\usepackage{geometry}
\usepackage{float}
\usepackage{url}
\usepackage[bottom]{footmisc}
\usepackage{hyperref}
\usepackage[nonumberlist, section=subsection]{glossaries}

\title{\textbf{Implementierung} \\ Cryptographics}
\author{}
\date{\today}

%Glossar-Befehle anschalten
\makeglossaries
% \newglossaryentry{identifier}{name={Name}, description={Description}}

\begin{document}

% The cover page.
\maketitle
\begin{table}[b]
  \begin{tabular}{| l | l | l |}
    \hline
    \textbf{Phase} & \textbf{Verantwortlicher} & \textbf{Email} \\ \hline
    Pflichtenheft & Matthias Jaenicke & matthias.jaenicke@student.kit.edu \\ \hline
    Entwurf & Matthias Plappert & undkc@student.kit.edu \\
            & Julien Duman & uncyc@student.kit.edu \\ \hline
    Implementierung & Christian Dreher & uaeef@student.kit.edu \\ \hline
    Qualitätssicherung & Wasilij Beskorovajnov & uajkm@student.kit.edu \\ \hline
    Präsentation & Aydin Tekin & aydin.tekin@student.kit.edu \\ \hline
    \end{tabular}
\end{table}
\thispagestyle{empty}
\newpage

% Table of contents page.
\tableofcontents
\newpage

% Start of the actual document.
\section{Einleitung}
In diesem Heft werden Änderungen zum Entwurf spezifiziert, die während der Implementierung entstanden sind.

\section{Grundsätzliche Änderungen}
Hier kommt rein wenn z.B. ein Package komplett weggefallen ist oder so

% Hier kommen alle neuen Klassen rein
\section{Neue Klassen}

  \subsection{Paket edu.kit.iks.Cryptographics}

  \subsection{Paket edu.kit.iks.CryptographicsLib}

  \subsection{Paket edu.kit.iks.Cryptographics.Vigenere}

  \subsection{Paket edu.kit.iks.Cryptographics.Vigenere.Demonstration}

  \subsection{Paket edu.kit.iks.Cryptographics.Vigenere.Experiment}

  \subsection{Paket edu.kit.iks.Cryptographics.Vigenere.Explanation}

  \subsection{Paket edu.kit.iks.Cryptographics.Caesar}
   \subsubsection{CryptoView}
          Diese Klasse besteht aus dem Programmcode, der aus den vorherigen Klassen CryptoView, das jetzt CryptoExperimentView heißt,
          und CipherDemoView, das jetzt CryptoDemonstrationView heißt, rausfaktorisiert wurde.
          
          Die beiden neuen Klassen CryptoExperimentView und CryptoDemonstrationView erben von dieser Klasse.\newline
           
    \textbf{Superklassen und Interfaces}
      \begin{itemize}
        \item  edu.kit.iks.CryptographicsLib.VisualizationView
      \end{itemize}
           
    \textbf{Methoden}
      \begin{itemize}
        \item protected CryptoView()\newline
              Konstruktor. Erzeugt die gemeinsamen Elemente von CryptoDemonstrationView und CryptoExperimentView.
        \item public void removeAlphabet()\newline
              Entfernt das Alphabet aus der Oberfläche. Public weil es oft in den Controllern
        \item protected void setupAlphabet()\newline
              Fügt das Alphabet der Oberfläche hinzu.
        \item public void removeKeyboard()\newline
              Entfernt die Tastatur für alphabetische Eingaben aus der Oberfläche.
        \item public void createKeyboard()\newline
              Fügt der Oberfläche eine Tastatur für alphabetische Eingaben hinzu.
        \item protected void setupInOutElements(char[] inputChars, int key)\newline
              Stellt die Buchstaben, die abzuarbeiten sind, voneinander getrennt in JLabels auf. 
              Auch die Textboxen, die für die Eingabe des Benutzers nötig sind werden hinzugefügt.
              Der Schlüssel wird separat neben den Buchstaben in einem JLabel dargestellt.
        \item protected void removeUserIOContainer()\newline
              Entfernt den Container für die JLabels mit den Buchstaben und die Textboxen aus der Oberlfäche.
        \item protected void removeExplanations()\newline
              Entfernt die Erklärungen aus der Oberlfäche.
	\item protected void setupExplanations(String explanations, final int flag,int yGrid, int xGrid, int widthGrid)\newline
	      Fügt der Oberfläche Erklärungen hinzu, die in dem String explanations enthalten sind.
	      Die restlichen Argumente sind werte für die Anordnung der Erklärungen in dem Layout.      
      \end{itemize}

  \subsection{Paket edu.kit.iks.Cryptographics.Caesar.Demonstration}
   \textbf{Keine neuen Klassen. Nur Änderungen an bestehenden.}
  \subsection{Paket edu.kit.iks.Cryptographics.Caesar.Experiment}
   \textbf{Keine neuen Klassen. Nur Änderungen an bestehenden.}
  \subsection{Paket edu.kit.iks.Cryptographics.DiffieHellman}
  
  \subsection{Paket edu.kit.iks.Cryptographics.DiffieHellman.Demonstration}

  \subsection{Paket edu.kit.iks.Cryptographics.DiffieHellman.Experiment}

% Hier kommen alle Änderungen an bestehenden Klassen rein
% (Dazu zählen auch Namensänderungen an Klassen. Schreibt dann einfach bei der Klasse hier
% rein: Klasse AlterName
% wurde zu NeuerName umbenannt
% etc...)
\section{Änderungen an Klassen}

  \subsection{Paket edu.kit.iks.Cryptographics}
    \subsubsection{Klasse HelpPopoverView}

    \subsubsection{Klasse Main}

    \subsubsection{Klasse MainController}

    \subsubsection{Klasse StartController}

    \subsubsection{Klasse TimelinePopoverView}

    \subsubsection{Klasse TimelineView}

    \subsubsection{Klasse VisualizationContainerController}

    \subsubsection{Klasse VisualizationContainerView}

    \subsubsection{Klasse WelcomeView}

  \subsection{Paket edu.kit.iks.CryptographicsLib}
    
  	\subsubsection{Klasse AbstractController}

	\subsubsection{Klasse AbstractVisualizationController}

	\subsubsection{Klasse AbstractVisualizationInfo}

	\subsubsection{Klasse AlphabetStripView}

	\subsubsection{Klasse CharacterFrequencyDiagramView}

	\subsubsection{Klasse ImageView}

	\subsubsection{Klasse InformationController}

	\subsubsection{Klasse InformationView}

	\subsubsection{Klasse KeyboardButton}

	\subsubsection{Klasse KeyboardView}

	\subsubsection{Klasse NumbersStripView}

	\subsubsection{Klasse PopoverView}

	\subsubsection{Klasse VisualizationButton}

	\subsubsection{Enum VisualizationDifficulty}

	\subsubsection{Klasse VisualizationInfoLoader}

	\subsubsection{Klasse VisualizationView}

  \subsection{Paket edu.kit.iks.Cryptographics.Vigenere}
    \subsubsection{Klasse AbstractController}

    \subsubsection{Klasse VigenereModel}

    \subsubsection{Klasse VigenereVisualizationInfo}

  \subsection{Paket edu.kit.iks.Cryptographics.Vigenere.Demonstration}
    \subsubsection{Klasse FirstDemonstrationController}

    \subsubsection{Klasse FirstDemonstrationView}

    \subsubsection{Klasse SecondDemonstrationController}

    \subsubsection{Klasse SecondDemonstrationView}

    \subsubsection{Klasse ThirdDemonstrationController}

    \subsubsection{Klasse ThirdDemonstrationView}

  \subsection{Paket edu.kit.iks.Cryptographics.Vigenere.Experiment}
    \subsubsection{Klasse FirstExperimentController}

    \subsubsection{Klasse FirstExperimentView}

    \subsubsection{Klasse SecondExperimentController}

    \subsubsection{Klasse SecondExperimentView}

  \subsection{Paket edu.kit.iks.Cryptographics.Vigenere.Explanation}
    \subsubsection{Klasse FirstExplanationController}

    \subsubsection{Klasse FirstExplanationView}

    \subsubsection{Klasse SecondExplanationController}

    \subsubsection{Klasse SecondExplanationView}

  \subsection{Paket edu.kit.iks.Cryptographics.Caesar}
  
  	\subsubsection{Klasse CaesarVisualizationInfo}
  	 \textbf{Keine Änderungen.}

 	\subsection{Paket edu.kit.iks.Cryptographics.Caesar.Demonstration}
          
          \subsubsection{Klasse IntroductionController}
            Die Methode animationProceed() wurde entfernt. Und durch die private Hilfsmethode proceedIntroduction() ersetzt.
            Diese erfüllt diesselbe Aufgabe und da sie zum Aufgabenbereich dieses Controllers gehört, ist ein globaler Zugriff
            unnötig.
       	  \subsubsection{Klasse IntroductionView}
            Die Methode animationStart() mitsamt ihrer privaten Hilfsfunktionen wurde in den IntroductionController ausgelagert.
            Die View dient als Container für Oberflächenelemente. Funktionalität braucht sie nicht.
	  \subsubsection{Klasse CipherDemoController}
            Diese Klasse heißt jetzt CryptoDemonstrationController.\newline
         
           \textbf{Änderungen an Methoden.}\newline
            proceedAnimation() wurde entfernt. Und durch die private Hilfsmethode demonstrate() ersetzt.
            Diese erfüllt diesselbe Aufgabe und da sie zum Aufgabenbereich dieses Controllers gehört, 
            ist ein globaler Zugriff unnötig.

	\subsubsection{Klasse CipherDemoView}
	   Diese Klasse heißt jetzt CryptoDemonstrationView.\newline
	   
	    \textbf{Änderungen an Superklassen und Interfaces.}\newline
	     Diese Klasse erbt nun von CryptoView(siehe das Kapitel 'Neue Klassen') und nicht mehr direkt von VisualizationView.\newline
	   
	    \textbf{Änderungen an Methoden.}\newline
	     Die Methode animationStart() mitsamt ihrer privaten Hilfsfunktionen wurde in den CryptoDemonstrationController ausgelagert.

  \subsection{Paket edu.kit.iks.Cryptographics.Caesar.Experiment}

	\subsubsection{Klasse CryptoController}
          Diese Klasse heißt jetzt CryptoExperimentController.\newline
          
          \textbf{Änderungen an Methoden.}\newline
           Es sind lediglich neue Hilfsmethoden dazugekommen.
	\subsubsection{Klasse CryptoModel}
	 An dieser Klasse wurde das Entwurfsmuster Singleton angewandt. Die vielseitige Verwendung dieser Klasse 
	 ohne den Bedarf einer mehrfachen Instanziierung ermöglichte dies.\newline 
	 
	  \textbf{Änderungen an Methoden.}\newline
            \begin{itemize}
             \item getInstance() wurde hinzugefügt\newline
                   Getter für die einzige Instanz dieser Klasse. Teil des Singleton Entwurfsmusters.
             \item Die Methode checkValidChar(char) wurde aus Redundanzgründen entfernt. 
                   Dieselbe Funktionalität bietet jetzt das verschlüsseln eines Zeichens
                   durch die Methode enc und dem vergleichen mit dem zu überprüfenden char.
             \item handleInput(String) wurde aus Gründen der Namenskonventionen umbenannt 
                   in isInputValid(String). 
             \item isInputStep() wurde entfernt, weil das Model jetzt nicht mehr an einen Controller gebunden ist
                   und keine Ahnung haben soll wie weit der User in der Visualisierung vorangekommen ist.
             \item Aus den gleichen Gründen wie davor fehlt auch der setter setInputStep() weg.
             \item decryptAndCheck(String) ist aus Redundanzgründen rausgenommen worden. Die Gründe sind identisch
                   mit der Entfernung von checkValidChar(char) nur bezieht es sich jetzt auf die Verarbeitung von
                   Strings und nicht chars, wie im oberen Fall.
             \item Weil die Klasse jetzt das Entwurfsmuster Singleton verwendet, ist der Konstruktor nicht mehr public,
                   sondern private.
             \item public String enc(int key, String text) wurde hinzugefügt.\newline
                   Diese Methode verschlüsselt caesarkonform einen beliebigen Text von beliebiger Länge, mit einem beliebigen Schlüssel.
                   Der Text kann auch Sonderzeichen enthalten. Sowie auch html tags.
             \item public String dec(int key, String cipher) wurde hinzugefügt.\newline
                   Ruft enc mit einem negativen Schlüssel auf. Bentötigt bei Entschlüsselung.
             \item private int generateRandomInt(int a, int b) wurde hinzugefügt.\newline
                   Erzeugt einen 'zufälligen' ganzzahligen Wert aus dem Intervall [a,b].
             \item public int generateKey() wurde hinzugefügt.\newline
                   Diese Methode erzeugt 'zufällig' einen Schlüssel für Caesar Ver-/Entschlüsselung.
                   Verwendet die methode private int generateRandomInt(int a, int b) mit Grenzen [1,26].
             \item public String genRandomBlamings() wurde hinzugefügt.\newline
                   Erzeugt 'zufällig' eine Negativmeldung zu den Interaktionen des Users
                   aus einem local Stringpool.  
             \item public String genRandomCipher(int key) wurde hinzugefügt.\newline
                   Verschlüsselt einen zufälligen Text mit dem gegebenen Schlüssel 
                   und gibt diesen aus.
             \item public String genRandomGrats() wurde hinzugefügt.\newline
                   Erzeugt 'zufällig' eine Positivmeldung zu den Interaktionen des Users
                   aus einem local Stringpool. 
             \item public String genRandomPlainSequence() wurde hinzugefügt.\newline
                   Erzeugt 'zufällig' ein kleines Wort aus einem local Stringpool zum verschlüsseln. 
             \item public String genRandomText() wurde hinzugefügt\newline
                   Erzeugt 'zufällig' einen größeren Text aus einem local Stringpool zum verschlüsseln.      
            \end{itemize}

	\subsubsection{Klasse CryptoView}
	 Diese Klasse wurde in CryptoExperimentView umbenannt.\newline
	 
	 \textbf{Änderungen an Superklassen und Interfaces.}\newline
	     Diese Klasse erbt nun von CryptoView(siehe das Kapitel 'Neue Klassen') und nicht mehr direkt von VisualizationView.\newline
	   
	 \textbf{Änderungen an Methoden.}\newline
	   \begin{itemize}
	     \item public void setupExperimentCore(char[] inputChars, int key) wurde hinzugefügt.\newline
	           Diese Methode erstellt alle für den Selbstversuch notwendigen Elemente und ordnet diese im Layout an.
             \item public void createNumpad()\newline
               Fügt der Oberfläche eine Tastatur für numerische Eingaben hinzu.
             \item public void removeNumpad()\newline
               Entfernt die Tastatur für numerische Eingaben aus der Oberlfäche.
           \end{itemize}
               
	\subsubsection{Klasse HistogramController}
          \textbf{Änderungen an Methoden.}\newline
           Es sind lediglich neue Hilfsmethoden dazugekommen.
	\subsubsection{Klasse HistogramView}
	 Die Methode startAnimation() wurde entfernt, da die View als Container für Oberflächenelemente dient. 
	 Funktionalität braucht sie nicht.

  \subsection{Paket edu.kit.iks.Cryptographics.DiffieHellman}
  
    \subsubsection{Klasse AbstractController}
    Diese Klasse wurde nicht gebraucht, ergo existiert sie nicht mehr

  	\subsubsection{Klasse Model}
    Singleton-Pattern wurde rückgängig gemacht, da die unterschiedlichen Zustände
    in einer Klasse behandelt werden und somit globale Variablen (also Singletons)
    unnötig werden

    \textbf{Änderungen an Methoden.}\newline
	   \begin{itemize}
           \item Konstruktor ist nicht mehr privat\newline
           \item public Model getInstance() wurde entfernt\newline
           \item public Color getPublicColor() wurde hinzugefügt\newline
           \item public void setPublicColor() wurde hinzugefügt\newline
           \item public void getAlicePublicColor() wurde entfernt\newline
           \item public void setAlicePublicColor() wurde entfernt\newline
           \item public Color getSharedColor() wurde entfernt\newline
           \item public void setSharedColor(Color setSharedColor) wurde entfernt\newline
           \item public void getBobPublicColor() wurde entfernt\newline
           \item public Color setBobPublicColor(Color bobPublicColor) wurde entfernt\newline
           \end{itemize}

	\subsubsection{Klasse DHVisualizationInfo}
    Hat sich nicht geändert

  \subsection{Paket edu.kit.iks.Cryptographics.DiffieHellman.Demonstration}

	\subsubsection{Klasse ExplainAimController}
    Hat sich entwurfstechnisch nicht verändert

	\subsubsection{Klasse ExplainAimView}
    \textbf{Änderungen an Methoden.}\newline
	   \begin{itemize}
           \item public ColorChannel getColorChannel() wurde hinzugefügt\newline
           \item public String getHelp() wurde hinzugefügt\newline
           \end{itemize}

	\subsubsection{Klasse OnewayController}
    Hat sich entwurfstechnisch nicht verändert

	\subsubsection{Klasse OnewayView}
    \textbf{Änderungen an Methoden.}\newline
	   \begin{itemize}
           \item public String getHelp() wurde hinzugefügt\newline
           \end{itemize}

	\subsubsection{Klasse ExplainKeyExchangeController}
    Diese Klasse wurde entfernt

	\subsubsection{Klasse ExplainKeyExchangeView}
    Diese Klasse wurde entfernt

	\subsubsection{Klasse AliceChooseSecretController}
    Diese Klasse wurde in DHDemoController umbenannt

	\subsubsection{Klasse AliceChooseSecretView}
    Diese Klasse wurde in DHDemoView umbenannt

	\subsubsection{Klasse BobChooseSecretController}
    Diese Klasse wurde entfernt

	\subsubsection{Klasse BobChooseSecretView}
    Diese Klasse wurde entfernt

	\subsubsection{Klasse MixColorController}
    Diese Klasse wurde entfernt

	\subsubsection{Klasse MixColorView}
    Diese Klasse wurde entfernt

	\subsubsection{Klasse ColorMix}
    Diese Klasse wurde neu hinzugefügt
    Ihre Aufgabe ist es als JPanel grafisch Farben zu mischen
    \textbf{Änderungen an Methoden.}\newline
	   \begin{itemize}
           \item public ColorMix(int circleSize, Dimension dimension) wurde hinzugefügt\newline
           \item public void mixColors(boolean mix, boolean repeat, final NextStepCallback cb) wurde hinzugefügt\newline
               Mischt grafisch zwei Farben zusammen
           \item public void seperateColors(boolean seperate, boolean repeat) wurde hinzugefügt\newline
               Trennt grafisch zwei Farben
           \item public void setEllipColor(int which, Color color) wurde hinzugefügt\newline
               Setzt erste oder zweite Farbe auf color
           \item public Color getMixedColor() wurde hinzugefügt\newline
               Gibt die gemischte Farbe zurück
           \item public boolean isComputeFinalMix() wurde hinzugefügt\newline
               Falls wahr, wird eine andere Formel zum Farbenmischen verwendet
           \item public void setComputeFinalMix(boolean computeFinalMix)\newline
               Setzt den Wert von computeFinalMix
           \end{itemize}

	\subsubsection{Klasse ColorChannel}
    Diese Klasse wurde neu hinzugefügt
    Ihre Aufgabe ist es als JPanel grafisch den
    Kommunikationskanal zwischen Alice, Bob und Eve
    zu vermitteln und Farben über diesen Kanal zu visuell
    zu verschicken.
    \textbf{Änderungen an Methoden.}\newline
	   \begin{itemize}
           \item public ColorChannel(Dimension d, int circleSize) wurde hinzugefügt\newline
               Der Konstruktor nimmt die Dimension und die Größe der Kreise als Parameter,
               und errechnet daraus wie die restlichen Elemente positioniert werden
           \item public void sendToBob(final NextStepCallback cb, final boolean keepFirst) wurde hinzugefügt\newline
               Schickt eine Farbkreis von Alice zu Bob, cb ist der Callback der aufgerufen wird, wenn dies
               erledigt wurde.
               Wenn keepFirst wahr ist, wird die Farbe neben Alice dauerhaft dargestellt
           \item public void sendToAlice(final NextStepCallback cb, final boolean keepFirst) wurde hinzugefügt\newline
               Tut das gleiche wie sendToBob, nur dass von Bob nach Alice die Farbe verschickt wird und
               wenn keepFirst den Wert wahr hat, wird die Farbe neben Bob visualisiert
           \item public void chooseColorToKeep(Color color, int who) wurde hinzugefügt\newline
               Ein Farbkreis mit der Farbe color wird neben who hinzugefügt also neben alice, bob oder eve
           \item public void stopTimer() wurde hinzugefügt\newline
               Stoppt den Timer. Muss aufgerufen werden wenn ColorChannel nicht mehr verwendet wird,
               da sonst der Timer weiterlaufen könnte
           \item public void loadView() wurde hinzugefügt\newline
               Tut im Grunde das gleiche wie im Konstruktor, nur dass diese Funktion dafür
               aufgerufen werden muss, wenn sich die Größe ändert
           \item public void choosePublicColor(Color color) wurde hinzugefügt\newline
               Setzt die öffentliche Farbe
           \item public void chooseAlicePrivateColor(Color color) wurde hinzugefügt\newline
               Setzt die private Farbe von Alice
           \item public void chooseBobPrivateColor(Color color) wurde hinzugefügt\newline
               Setzt die private Farbe von Bob
           \item public void mixAlicePrivatePublic() wurde hinzugefügt\newline
               Mischt die private Farbe von Alice und die öffentliche Farbe
               und speichert diese
           \item public void mixBobPrivatePublic() wurde hinzugefügt\newline
               Mischt die private Farbe von Bob und die öffentliche Farbe
               und speichert diese
           \item public void sendPublicColor() wurde hinzugefügt\newline
               Sendet die öffentliche Farbe von Alice zu Bob
           \item public void sendAliceMixedColorToBob(NextStepCallback cb) wurde hinzugefügt\newline
               Sendet die Mixtur von Alice zu Bob, und ruft cb auf wenn dies geschehen ist
           \item public void sendBobMixedColorToAlice(NextStepCallback cb) wurde hinzugefügt\newline
               Sendet die Mixtur von Bob zu Alice , und ruft cb auf wenn dies geschehen ist
           \item public void mixAliceFinalSecret(NextStepCallback cb) wurde hinzugefügt\newline
               Mischt das gemeinsame Geheimnis zusammen und ruft cb auf, wenn dies geschehen ist
           \item public void mixBobFinalSecret(NextStepCallback cb) wurde hinzugefügt\newline
               Mischt das gemeinsame Geheimnis zusammen und ruft cb auf, wenn dies geschehen ist
           \item public boolean isRepeat() wurde hinzugefügt\newline
               Wahr wenn das Senden in einer periodischen Animation laufen soll
           \item public void setRepeat(boolean repeat) wurde hinzugefügt\newline
               Setzt den repeat Wert
           \item 
           \end{itemize}

	\subsubsection{Klasse ColorChooser}
    Diese Klasse wurde neu hinzugefügt

	\subsubsection{Klasse DemoOneWayController}
    Diese Klasse wurde neu hinzugefügt

	\subsubsection{Klasse DemoOneWayView}
    Diese Klasse wurde neu hinzugefügt

	\subsubsection{Klasse Ellipse2DwithColor}
    Diese Klasse wurde neu hinzugefügt

	\subsubsection{Interface NextStepCallback}
    Diese Interface wurde neu hinzugefügt

  \subsection{Paket edu.kit.iks.Cryptographics.DiffieHellman.Experiment}

	\subsubsection{Klasse YourTurnController}

	\subsubsection{Klasse YourTurnView}

	\subsubsection{Klasse ChoosePublicColorController}

	\subsubsection{Klasse ChoosePublicColorView}

	\subsubsection{Klasse SendColorController}

	\subsubsection{Klasse SendColorView}

	\subsubsection{Klasse ChooseSecretColorController}

	\subsubsection{Klasse ChooseSecretColorView}

	\subsubsection{Klasse SendRightColorController}

	\subsubsection{Klasse SendRightColorView}

	\subsubsection{Klasse MixFinalSecretController}

	\subsubsection{Klasse MixFinalSecretView}
	
 \restoregeometry

\glsaddall
\printglossary[numberedsection, style=altlist]

\end{document}
