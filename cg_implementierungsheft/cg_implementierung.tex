\documentclass{article}

\usepackage[utf8]{inputenc}
\usepackage[ngerman]{babel}
\usepackage[ngerman]{translator}
\usepackage[T1]{fontenc}
\usepackage{enumitem}
\usepackage{graphicx}
\usepackage{geometry}
\usepackage{float}
\usepackage{url}
\usepackage[bottom]{footmisc}
\usepackage{hyperref}
\usepackage[nonumberlist, section=subsection]{glossaries}

\title{\textbf{Implementierung} \\ Cryptographics}
\author{}
\date{\today}

%Glossar-Befehle anschalten
\makeglossaries
% \newglossaryentry{identifier}{name={Name}, description={Description}}

\begin{document}

% The cover page.
\maketitle
\begin{table}[b]
  \begin{tabular}{| l | l | l |}
    \hline
    \textbf{Phase} & \textbf{Verantwortlicher} & \textbf{Email} \\ \hline
    Pflichtenheft & Matthias Jaenicke & matthias.jaenicke@student.kit.edu \\ \hline
    Entwurf & Matthias Plappert & undkc@student.kit.edu \\
            & Julien Duman & uncyc@student.kit.edu \\ \hline
    Implementierung & Christian Dreher & uaeef@student.kit.edu \\ \hline
    Qualitätssicherung & Wasilij Beskorovajnov & uajkm@student.kit.edu \\ \hline
    Präsentation & Aydin Tekin & aydin.tekin@student.kit.edu \\ \hline
    \end{tabular}
\end{table}
\thispagestyle{empty}
\newpage

% Table of contents page.
\tableofcontents
\newpage

% Start of the actual document.
\section{Einleitung}
In diesem Heft werden Änderungen zum Entwurf spezifiziert, die während der Implementierung entstanden sind.

\section{Grundsätzliche Änderungen}
Hier kommt rein wenn z.B. ein Package komplett weggefallen ist oder so

% Hier kommen alle neuen Klassen rein
\section{Neue Klassen}

  \subsection{Paket edu.kit.iks.Cryptographics}
    \subsubsection{IdlePopoverView}
    Zur Zeit des Entwurfs war noch unklar, wie wir optimisch mit einem inaktiven Benutzer umgehen.
    Nach einigen Tests haben wir uns dazu entschieden, das existierende Popover-System zu verwenden. Daher wurde
    die Klasse IdlePopoverView nötig. Diese stellt einen kurzen Informationstext dar, der dem Benutzer erklärt,
    dass sich das Programm in kürze zurücksetzen wird. Die verbleibende Zeit wird in Sekunden dargestellt. Außerdem
    wird ein Button dargestellt, mit dem das Zurücksetzen verhindert werden kann.\newline

    \textbf{Superklassen und Interfaces}
      \begin{itemize}
        \item edu.kit.iks.CryptographicsLib.PopoverView
      \end{itemize}
           
    \textbf{Methoden}
      \begin{itemize}
        \item public IdlePopoverView(long initialTime)\newline
              Konstruktor. Erzeugt eine neue Instanz. Der Countdown wird bei initialTime in Millisekunden
              gestartet.
        \item public JButton getContinueButton()\newline
              Gibt den Fortfahren-Button zurück. Ein Controller kann mithilfe eines ActionListeners
              das Zurücksetzen beim Anklicken verhindern.
      \end{itemize}

  \subsection{Paket edu.kit.iks.CryptographicsLib}
    \subsubsection{Configuration}
    Während der Implementierung hat sich gezeigt, das an einigen Stellen Konstanten verwendet werden,
    beispielsweise die Zeit in Sekunden, nachdem der Benutzer als inaktiv gilt. Um diese global ändern zu können,
    wurde die Klasse Configuration angelegt. Die Werte stammen aus einer XML-Datei. Falls diese nicht existiert,
    werden voreingestellte Default-Werte verwendet. Die Klasse Configuration verwendet das Singleton-Entwurfsmuster.\newline

    \textbf{Superklassen und Interfaces}
      \begin{itemize}
        \item \textit{keine}
      \end{itemize}
           
    \textbf{Methoden}
      \begin{itemize}
        \item public static getInstance()\newline
              Gibt die Singleton-Instanz zurück bzw. erzeugt diese.
        \item public int getIdleTimeout()\newline
              Gibt die Zeit in Millisekunden zurück nach der der Benutzer als inaktiv gilt.
        \item public int getResetTimeout()\newline
              Gibt die Zeit in Millisekunden zurück nach der sich das Programm in den Grundzustand
              zurücksetzt zurück. Dieser Timeout beginnt erst, nachdem der Benutzer als inaktiv gilt.
        \item public boolean isDebugModeEnabled()\newline
              Deutet an, ob das Programm im Debug-Modus ausgeführt werden soll.
        \item public boolean isMouseCursorEnabled()\newline
              Deutet an, ob der Mauszeiger angezeigt werden soll. Falls das Programm nicht im Debug-Modus
              ausgeführt wird, gibt diese Methode immer false zurück.
        \item public boolean isLookAndFeelEnabled()\newline
              Deutet an, ob das Look \& Feel verwendet werden soll. Falls das Programm nicht im Debug-Modus
              ausgeführt wird, gibt diese Methode immer true zurück.
        \item public boolean isFullscreenModeEnabled()\newline
              Deutet an, ob das Programm im Vollbildmodus ausgeführt werden soll. Falls das Programm nicht
              im Debug-Modus ausgeführt wird, gibt diese Methode immer true zurück.
        \item public String getLanguageCode()\newline
              Gibt den für die Lokalisierung notwendigen Sprachcode zurück. Dieser kann beispielsweise ``de''
              oder ``en'' sein.
        \item public I18n getI18n(Class className)\newline
              Gibt die für die gegebene Klasse geeignete I18n-Instanz zurück. Diese wird zur Lokalisierung
              verwendet.
      \end{itemize}

  \subsection{Paket edu.kit.iks.Cryptographics.Vigenere}

  \subsection{Paket edu.kit.iks.Cryptographics.Vigenere.Demonstration}

  \subsection{Paket edu.kit.iks.Cryptographics.Vigenere.Experiment}

  \subsection{Paket edu.kit.iks.Cryptographics.Vigenere.Explanation}

  \subsection{Paket edu.kit.iks.Cryptographics.Caesar}
   \subsubsection{CryptoView}
          Diese Klasse besteht aus dem Programmcode, der aus den vorherigen Klassen CryptoView, das jetzt CryptoExperimentView heißt,
          und CipherDemoView, das jetzt CryptoDemonstrationView heißt, rausfaktorisiert wurde.
          
          Die beiden neuen Klassen CryptoExperimentView und CryptoDemonstrationView erben von dieser Klasse.\newline
           
    \textbf{Superklassen und Interfaces}
      \begin{itemize}
        \item  edu.kit.iks.CryptographicsLib.VisualizationView
      \end{itemize}
           
    \textbf{Methoden}
      \begin{itemize}
        \item protected CryptoView()\newline
              Konstruktor. Erzeugt die gemeinsamen Elemente von CryptoDemonstrationView und CryptoExperimentView.
        \item public void removeAlphabet()\newline
              Entfernt das Alphabet aus der Oberfläche. Public weil es oft in den Controllern
        \item protected void setupAlphabet()\newline
              Fügt das Alphabet der Oberfläche hinzu.
        \item public void removeKeyboard()\newline
              Entfernt die Tastatur für alphabetische Eingaben aus der Oberfläche.
        \item public void createKeyboard()\newline
              Fügt der Oberfläche eine Tastatur für alphabetische Eingaben hinzu.
        \item protected void setupInOutElements(char[] inputChars, int key)\newline
              Stellt die Buchstaben, die abzuarbeiten sind, voneinander getrennt in JLabels auf. 
              Auch die Textboxen, die für die Eingabe des Benutzers nötig sind werden hinzugefügt.
              Der Schlüssel wird separat neben den Buchstaben in einem JLabel dargestellt.
        \item protected void removeUserIOContainer()\newline
              Entfernt den Container für die JLabels mit den Buchstaben und die Textboxen aus der Oberlfäche.
        \item protected void removeExplanations()\newline
              Entfernt die Erklärungen aus der Oberlfäche.
	\item protected void setupExplanations(String explanations, final int flag,int yGrid, int xGrid, int widthGrid)\newline
	      Fügt der Oberfläche Erklärungen hinzu, die in dem String explanations enthalten sind.
	      Die restlichen Argumente sind werte für die Anordnung der Erklärungen in dem Layout.      
      \end{itemize}

  \subsection{Paket edu.kit.iks.Cryptographics.Caesar.Demonstration}
   \textbf{Keine neuen Klassen. Nur Änderungen an bestehenden.}
  \subsection{Paket edu.kit.iks.Cryptographics.Caesar.Experiment}
   \textbf{Keine neuen Klassen. Nur Änderungen an bestehenden.}
  \subsection{Paket edu.kit.iks.Cryptographics.DiffieHellman}
  
  \subsection{Paket edu.kit.iks.Cryptographics.DiffieHellman.Demonstration}

  \subsection{Paket edu.kit.iks.Cryptographics.DiffieHellman.Experiment}

% Hier kommen alle Änderungen an bestehenden Klassen rein
% (Dazu zählen auch Namensänderungen an Klassen. Schreibt dann einfach bei der Klasse hier
% rein: Klasse AlterName
% wurde zu NeuerName umbenannt
% etc...)
\section{Änderungen an Klassen}

  \subsection{Paket edu.kit.iks.Cryptographics}
    \subsubsection{HelpPopoverView}
      Keine Änderungen.

    \subsubsection{Main}
      Keine Änderungen.

    \subsubsection{MainController}
      In der Klasse MainController wurden Getter für das JFrame, sowie der StartController
      und VisualizationContainerController hinzugefügt.\newline

      \textbf{Änderungen an Methoden}
      \begin{itemize}
        \item public JFrame getJFrame()\newline
              Gibt das JFrame zurück.
        \item public StartController getStartController()\newline
              Gibt den StartController zurück.
        \item public VisualizationContainerController getVisualizationContainerController()\newline
              Gibt den VisualizationContainerController zurück.
      \end{itemize}

    \subsubsection{StartController}
      StartController hat eine minimale Änderung an der Methodensignatur erfahren.\newline

      \textbf{Änderungen an Methoden}
      \begin{itemize}
        \item public void presentPopoverAction(AbstractVisualizationInfo visualizationInfo, JComponent sender)\newline
              Die Methode erwartet nun einen zusätzlichen Parameter sender. Dieser gibt an, welches JComponent
              die Aktion ausgelöst hat, um den Popover korrekt darstellen zu können.
      \end{itemize}

    \subsubsection{TimelinePopoverView}
      Keine Änderungen.

    \subsubsection{TimelineView}
      Keine Änderungen.

    \subsubsection{VisualizationContainerController}
      Der VisualizationContainerController unterstützt nun die Darstellung der IdlePopoverView.\newline

      \textbf{Änderungen an Methoden}
      \begin{itemize}
        \item public void presentIdlePopover()\newline
              Zeigt die IdlePopoverView an.
        \item public void dismissIdlePopover()\newline
              Verbirgt die IdlePopoverView.
      \end{itemize}

    \subsubsection{VisualizationContainerView}
      Die VisualizationContainerView besitzt nun eine Content-View. Diese dient als Container für die Visualisierung
      und erlaubt es uns, den Inhalt der Visualisierung korrekt zu positionieren.\newline

      \textbf{Änderungen an Methoden}
      \begin{itemize}
        \item public JPanel getContentView()\newline
              Gibt die Content-View zurück.
      \end{itemize}

    \subsubsection{WelcomeView}
      Keine Änderungen.

  \subsection{Paket edu.kit.iks.CryptographicsLib}
    
  	\subsubsection{Klasse AbstractController}

	\subsubsection{Klasse AbstractVisualizationController}

	\subsubsection{Klasse AbstractVisualizationInfo}

	\subsubsection{Klasse AlphabetStripView}

	\subsubsection{Klasse CharacterFrequencyDiagramView}

	\subsubsection{Klasse ImageView}

	\subsubsection{Klasse InformationController}

	\subsubsection{Klasse InformationView}

	\subsubsection{Klasse KeyboardButton}

	\subsubsection{Klasse KeyboardView}

	\subsubsection{Klasse NumbersStripView}

	\subsubsection{Klasse PopoverView}

	\subsubsection{Klasse VisualizationButton}

	\subsubsection{Enum VisualizationDifficulty}

	\subsubsection{Klasse VisualizationInfoLoader}

	\subsubsection{Klasse VisualizationView}

  \subsection{Paket edu.kit.iks.Cryptographics.Vigenere}
    \subsubsection{Klasse AbstractController}

    \subsubsection{Klasse VigenereModel}

    \subsubsection{Klasse VigenereVisualizationInfo}

  \subsection{Paket edu.kit.iks.Cryptographics.Vigenere.Demonstration}
    \subsubsection{Klasse FirstDemonstrationController}

    \subsubsection{Klasse FirstDemonstrationView}

    \subsubsection{Klasse SecondDemonstrationController}

    \subsubsection{Klasse SecondDemonstrationView}

    \subsubsection{Klasse ThirdDemonstrationController}

    \subsubsection{Klasse ThirdDemonstrationView}

  \subsection{Paket edu.kit.iks.Cryptographics.Vigenere.Experiment}
    \subsubsection{Klasse FirstExperimentController}

    \subsubsection{Klasse FirstExperimentView}

    \subsubsection{Klasse SecondExperimentController}

    \subsubsection{Klasse SecondExperimentView}

  \subsection{Paket edu.kit.iks.Cryptographics.Vigenere.Explanation}
    \subsubsection{Klasse FirstExplanationController}

    \subsubsection{Klasse FirstExplanationView}

    \subsubsection{Klasse SecondExplanationController}

    \subsubsection{Klasse SecondExplanationView}

  \subsection{Paket edu.kit.iks.Cryptographics.Caesar}
  
  	\subsubsection{Klasse CaesarVisualizationInfo}
  	 \textbf{Keine Änderungen.}

 	\subsection{Paket edu.kit.iks.Cryptographics.Caesar.Demonstration}
          
          \subsubsection{Klasse IntroductionController}
            Die Methode animationProceed() wurde entfernt. Und durch die private Hilfsmethode proceedIntroduction() ersetzt.
            Diese erfüllt diesselbe Aufgabe und da sie zum Aufgabenbereich dieses Controllers gehört, ist ein globaler Zugriff
            unnötig.
       	  \subsubsection{Klasse IntroductionView}
            Die Methode animationStart() mitsamt ihrer privaten Hilfsfunktionen wurde in den IntroductionController ausgelagert.
            Die View dient als Container für Oberflächenelemente. Funktionalität braucht sie nicht.
	  \subsubsection{Klasse CipherDemoController}
            Diese Klasse heißt jetzt CryptoDemonstrationController.\newline
         
           \textbf{Änderungen an Methoden.}\newline
            proceedAnimation() wurde entfernt. Und durch die private Hilfsmethode demonstrate() ersetzt.
            Diese erfüllt diesselbe Aufgabe und da sie zum Aufgabenbereich dieses Controllers gehört, 
            ist ein globaler Zugriff unnötig.

	\subsubsection{Klasse CipherDemoView}
	   Diese Klasse heißt jetzt CryptoDemonstrationView.\newline
	   
	    \textbf{Änderungen an Superklassen und Interfaces.}\newline
	     Diese Klasse erbt nun von CryptoView(siehe das Kapitel 'Neue Klassen') und nicht mehr direkt von VisualizationView.\newline
	   
	    \textbf{Änderungen an Methoden.}\newline
	     Die Methode animationStart() mitsamt ihrer privaten Hilfsfunktionen wurde in den CryptoDemonstrationController ausgelagert.

  \subsection{Paket edu.kit.iks.Cryptographics.Caesar.Experiment}

	\subsubsection{Klasse CryptoController}
          Diese Klasse heißt jetzt CryptoExperimentController.\newline
          
          \textbf{Änderungen an Methoden.}\newline
           Es sind lediglich neue Hilfsmethoden dazugekommen.
	\subsubsection{Klasse CryptoModel}
	 An dieser Klasse wurde das Entwurfsmuster Singleton angewandt. Die vielseitige Verwendung dieser Klasse 
	 ohne den Bedarf einer mehrfachen Instanziierung ermöglichte dies.\newline 
	 
	  \textbf{Änderungen an Methoden.}\newline
            \begin{itemize}
             \item getInstance() wurde hinzugefügt\newline
                   Getter für die einzige Instanz dieser Klasse. Teil des Singleton Entwurfsmusters.
             \item Die Methode checkValidChar(char) wurde aus Redundanzgründen entfernt. 
                   Dieselbe Funktionalität bietet jetzt das verschlüsseln eines Zeichens
                   durch die Methode enc und dem vergleichen mit dem zu überprüfenden char.
             \item handleInput(String) wurde aus Gründen der Namenskonventionen umbenannt 
                   in isInputValid(String). 
             \item isInputStep() wurde entfernt, weil das Model jetzt nicht mehr an einen Controller gebunden ist
                   und keine Ahnung haben soll wie weit der User in der Visualisierung vorangekommen ist.
             \item Aus den gleichen Gründen wie davor fehlt auch der setter setInputStep() weg.
             \item decryptAndCheck(String) ist aus Redundanzgründen rausgenommen worden. Die Gründe sind identisch
                   mit der Entfernung von checkValidChar(char) nur bezieht es sich jetzt auf die Verarbeitung von
                   Strings und nicht chars, wie im oberen Fall.
             \item Weil die Klasse jetzt das Entwurfsmuster Singleton verwendet, ist der Konstruktor nicht mehr public,
                   sondern private.
             \item public String enc(int key, String text) wurde hinzugefügt.\newline
                   Diese Methode verschlüsselt caesarkonform einen beliebigen Text von beliebiger Länge, mit einem beliebigen Schlüssel.
                   Der Text kann auch Sonderzeichen enthalten. Sowie auch html tags.
             \item public String dec(int key, String cipher) wurde hinzugefügt.\newline
                   Ruft enc mit einem negativen Schlüssel auf. Bentötigt bei Entschlüsselung.
             \item private int generateRandomInt(int a, int b) wurde hinzugefügt.\newline
                   Erzeugt einen 'zufälligen' ganzzahligen Wert aus dem Intervall [a,b].
             \item public int generateKey() wurde hinzugefügt.\newline
                   Diese Methode erzeugt 'zufällig' einen Schlüssel für Caesar Ver-/Entschlüsselung.
                   Verwendet die methode private int generateRandomInt(int a, int b) mit Grenzen [1,26].
             \item public String genRandomBlamings() wurde hinzugefügt.\newline
                   Erzeugt 'zufällig' eine Negativmeldung zu den Interaktionen des Users
                   aus einem local Stringpool.  
             \item public String genRandomCipher(int key) wurde hinzugefügt.\newline
                   Verschlüsselt einen zufälligen Text mit dem gegebenen Schlüssel 
                   und gibt diesen aus.
             \item public String genRandomGrats() wurde hinzugefügt.\newline
                   Erzeugt 'zufällig' eine Positivmeldung zu den Interaktionen des Users
                   aus einem local Stringpool. 
             \item public String genRandomPlainSequence() wurde hinzugefügt.\newline
                   Erzeugt 'zufällig' ein kleines Wort aus einem local Stringpool zum verschlüsseln. 
             \item public String genRandomText() wurde hinzugefügt\newline
                   Erzeugt 'zufällig' einen größeren Text aus einem local Stringpool zum verschlüsseln.      
            \end{itemize}

	\subsubsection{Klasse CryptoView}
	 Diese Klasse wurde in CryptoExperimentView umbenannt.\newline
	 
	 \textbf{Änderungen an Superklassen und Interfaces.}\newline
	     Diese Klasse erbt nun von CryptoView(siehe das Kapitel 'Neue Klassen') und nicht mehr direkt von VisualizationView.\newline
	   
	 \textbf{Änderungen an Methoden.}\newline
	   \begin{itemize}
	     \item public void setupExperimentCore(char[] inputChars, int key) wurde hinzugefügt.\newline
	           Diese Methode erstellt alle für den Selbstversuch notwendigen Elemente und ordnet diese im Layout an.
             \item public void createNumpad()\newline
               Fügt der Oberfläche eine Tastatur für numerische Eingaben hinzu.
             \item public void removeNumpad()\newline
               Entfernt die Tastatur für numerische Eingaben aus der Oberlfäche.
           \end{itemize}
               
	\subsubsection{Klasse HistogramController}
          \textbf{Änderungen an Methoden.}\newline
           Es sind lediglich neue Hilfsmethoden dazugekommen.
	\subsubsection{Klasse HistogramView}
	 Die Methode startAnimation() wurde entfernt, da die View als Container für Oberflächenelemente dient. 
	 Funktionalität braucht sie nicht.

  \subsection{Paket edu.kit.iks.Cryptographics.DiffieHellman}
  
    \subsubsection{Klasse AbstractController}

  	\subsubsection{Klasse Model}

	\subsubsection{Klasse DHVisualizationInfo}

  \subsection{Paket edu.kit.iks.Cryptographics.DiffieHellman.Demonstration}

	\subsubsection{Klasse ExplainAimController}

	\subsubsection{Klasse ExplainAimView}

	\subsubsection{Klasse OnewayController}

	\subsubsection{Klasse OnewayView}

	\subsubsection{Klasse ExplainKeyExchangeController}

	\subsubsection{Klasse ExplainKeyExchangeView}

	\subsubsection{Klasse AliceChooseSecretController}

	\subsubsection{Klasse AliceChooseSecretView}

	\subsubsection{Klasse BobChooseSecretController}

	\subsubsection{Klasse BobChooseSecretView}

	\subsubsection{Klasse MixColorController}

	\subsubsection{Klasse MixColorView}

  \subsection{Paket edu.kit.iks.Cryptographics.DiffieHellman.Experiment}

	\subsubsection{Klasse YourTurnController}

	\subsubsection{Klasse YourTurnView}

	\subsubsection{Klasse ChoosePublicColorController}

	\subsubsection{Klasse ChoosePublicColorView}

	\subsubsection{Klasse SendColorController}

	\subsubsection{Klasse SendColorView}

	\subsubsection{Klasse ChooseSecretColorController}

	\subsubsection{Klasse ChooseSecretColorView}

	\subsubsection{Klasse SendRightColorController}

	\subsubsection{Klasse SendRightColorView}

	\subsubsection{Klasse MixFinalSecretController}

	\subsubsection{Klasse MixFinalSecretView}
	
 \restoregeometry

\glsaddall
\printglossary[numberedsection, style=altlist]

\end{document}
