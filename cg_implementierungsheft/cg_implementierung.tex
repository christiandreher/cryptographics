\documentclass{article}

\usepackage[utf8]{inputenc}
\usepackage[ngerman]{babel}
\usepackage[ngerman]{translator}
\usepackage[T1]{fontenc}
\usepackage{enumitem}
\usepackage{graphicx}
\usepackage{geometry}
\usepackage{float}
\usepackage{url}
\usepackage[bottom]{footmisc}
\usepackage{hyperref}
\usepackage[nonumberlist, section=subsection]{glossaries}

\title{\textbf{Implementierung} \\ Cryptographics}
\author{}
\date{\today}

%Glossar-Befehle anschalten
\makeglossaries
% \newglossaryentry{identifier}{name={Name}, description={Description}}

\begin{document}

% The cover page.
\maketitle
\begin{table}[b]
  \begin{tabular}{| l | l | l |}
    \hline
    \textbf{Phase} & \textbf{Verantwortlicher} & \textbf{Email} \\ \hline
    Pflichtenheft & Matthias Jaenicke & matthias.jaenicke@student.kit.edu \\ \hline
    Entwurf & Matthias Plappert & undkc@student.kit.edu \\
            & Julien Duman & uncyc@student.kit.edu \\ \hline
    Implementierung & Christian Dreher & uaeef@student.kit.edu \\ \hline
    Qualitätssicherung & Wasilij Beskorovajnov & uajkm@student.kit.edu \\ \hline
    Präsentation & Aydin Tekin & aydin.tekin@student.kit.edu \\ \hline
    \end{tabular}
\end{table}
\thispagestyle{empty}
\newpage

% Table of contents page.
\tableofcontents
\newpage

% Start of the actual document.
\section{Einleitung}
In diesem Heft werden Änderungen zum Entwurf spezifiziert, die während der Implementierung entstanden sind.

Der Entwurf des Rahmenwerks hat sich während der Implementierung nur geringfügig geändert. Grund für die Änderungen
waren vor allem Implementierungsdetails von Swing sowie leicht geänderte Anforderungen an die geteilten UI-Komponenten.
Die grundlegende Kommunikation zwischen den einzelnen Komponenten hat sich nicht geändert. Alle vorgesehenen Aufgaben
konnten problemlos erfüllt werden, und auch die oben genannten Änderungen ließen sich problemlos durchführen.

Der Entwurf der einzelnen Verfahren hat sich mitunter stark verändert. Dies war bereits während der Entwurfsphase
absehbar (siehe Entwurfsheft). Der Grund hierfür liegt vor allem darin, dass sich die genaue Aufteilung der UI in die
einzelnen Schritte sowie die Präsentation durch zahlreiche Iterationen und Selbsttests ergeben hat. Auch dies war
problemlos möglich, da das Rahmenwerk den einzelnen Verfahren große Flexibilität in ihrer gewählten Darstellung
erlaubt. 

\section{Grundsätzliche Änderungen}
Während der Implementierung sind keine grundsätzlichen Änderungen vorgenommen worden.

% Hier kommen alle neuen Klassen rein
\section{Neue Klassen}

  \subsection{Paket edu.kit.iks.Cryptographics}
    \subsubsection{IdlePopoverView}
    Zur Zeit des Entwurfs war noch unklar, wie wir optimisch mit einem inaktiven Benutzer umgehen.
    Nach einigen Tests haben wir uns dazu entschieden, das existierende Popover-System zu verwenden. Daher wurde
    die Klasse IdlePopoverView nötig. Diese stellt einen kurzen Informationstext dar, der dem Benutzer erklärt,
    dass sich das Programm in kürze zurücksetzen wird. Die verbleibende Zeit wird in Sekunden dargestellt. Außerdem
    wird ein Button dargestellt, mit dem das Zurücksetzen verhindert werden kann.\newline

    \textbf{Superklassen und Interfaces}
      \begin{itemize}
        \item edu.kit.iks.CryptographicsLib.PopoverView
      \end{itemize}
           
    \textbf{Methoden}
      \begin{itemize}
        \item public IdlePopoverView(long initialTime)\newline
              Konstruktor. Erzeugt eine neue Instanz. Der Countdown wird bei initialTime in Millisekunden
              gestartet.
        \item public JButton getContinueButton()\newline
              Gibt den Fortfahren-Button zurück. Ein Controller kann mithilfe eines ActionListeners
              das Zurücksetzen beim Anklicken verhindern.
      \end{itemize}

  \subsection{Paket edu.kit.iks.CryptographicsLib}
    \subsubsection{Configuration}
    Während der Implementierung hat sich gezeigt, das an einigen Stellen Konstanten verwendet werden,
    beispielsweise die Zeit in Sekunden, nachdem der Benutzer als inaktiv gilt. Um diese global ändern zu können,
    wurde die Klasse Configuration angelegt. Die Werte stammen aus einer XML-Datei. Falls diese nicht existiert,
    werden voreingestellte Default-Werte verwendet. Die Klasse Configuration verwendet das Singleton-Entwurfsmuster.\newline

    \textbf{Superklassen und Interfaces}
      \begin{itemize}
        \item \textit{keine}
      \end{itemize}
           
    \textbf{Methoden}
      \begin{itemize}
        \item public static getInstance()\newline
              Gibt die Singleton-Instanz zurück bzw. erzeugt diese.
        \item public int getIdleTimeout()\newline
              Gibt die Zeit in Millisekunden zurück nach der der Benutzer als inaktiv gilt.
        \item public int getResetTimeout()\newline
              Gibt die Zeit in Millisekunden zurück nach der sich das Programm in den Grundzustand
              zurücksetzt zurück. Dieser Timeout beginnt erst, nachdem der Benutzer als inaktiv gilt.
        \item public boolean isDebugModeEnabled()\newline
              Deutet an, ob das Programm im Debug-Modus ausgeführt werden soll.
        \item public boolean isMouseCursorEnabled()\newline
              Deutet an, ob der Mauszeiger angezeigt werden soll. Falls das Programm nicht im Debug-Modus
              ausgeführt wird, gibt diese Methode immer false zurück.
        \item public boolean isLookAndFeelEnabled()\newline
              Deutet an, ob das Look \& Feel verwendet werden soll. Falls das Programm nicht im Debug-Modus
              ausgeführt wird, gibt diese Methode immer true zurück.
        \item public boolean isFullscreenModeEnabled()\newline
              Deutet an, ob das Programm im Vollbildmodus ausgeführt werden soll. Falls das Programm nicht
              im Debug-Modus ausgeführt wird, gibt diese Methode immer true zurück.
        \item public String getLanguageCode()\newline
              Gibt den für die Lokalisierung notwendigen Sprachcode zurück. Dieser kann beispielsweise ``de''
              oder ``en'' sein.
        \item public I18n getI18n(Class className)\newline
              Gibt die für die gegebene Klasse geeignete I18n-Instanz zurück. Diese wird zur Lokalisierung
              verwendet.
      \end{itemize}


    \subsubsection{Logger}
    Diese Klasse dient dem Zweck, während der Implementierung eine einheitliche Debuggingschnittstelle
    zu bieten, mögliche Exceptions durch Cryptographics abzufangen und in eine error.log-Datei zu schreiben,
    und Nutzerstatistiken in statistics.csv zu speichern. Sämtliche Methoden innerhalb dieser Klasse
    sind statisch und ohne Instanziierung zugreifbar.\newline

    \textbf{Superklassen und Interfaces}
      \begin{itemize}
        \item \textit{keine}
      \end{itemize}
           
    \textbf{Methoden}
      \begin{itemize}
        \item public static void l(String logEntry)\newline
              l steht für Log. Diese Methode loggt den übergebenen logEntry mit dem Zeitpunkt
              des Aufrufs der l()-Methode in statistics.csv zur späteren Auswertung von Statistiken.
              Während des Debugmodus' werden die Ausgaben auch in die Konsole geschrieben.
		\item public static void d(String classID, String method, String debugText) \newline
			  d steht für Debug. Diese Methode dient als einheitliche Debugausgabenschnittstelle, 
			  welche die Debugausgaben nur dann in die Konsole schreibt, wenn der Debugmodus aktiviert ist.
			  classID und method stehen dabei für die Klasse und die Methode, von der der Log aufgerufen wurde,
			  und der debugText ist der Text, der ausgegeben werden soll.
		\item public static void e(Exception exception) \newline
			  e steht für Exception. Diese Methode wird innerhalb eines catch-Blocks aufgerufen, und schreibt
			  die Exceptionmeldung sowie den Stacktrace der Exception in die error.log-Datei.
      \end{itemize}

    \subsubsection{MouseClickListener}
    Einfacher Bequemlichkeitslistener für Mausklicks. Er implementiert sämtliche Methoden von 
    MouseListener, jedoch sind alle bis auf clicked() nur Stummel.\newline

    \textbf{Superklassen und Interfaces}
      \begin{itemize}
        \item MouseListener
      \end{itemize}
           
    \textbf{Methoden}
      \begin{itemize}
        \item abstract public void clicked(MouseEvent event)\newline
        	Muss implementiert werden, um zu definieren, was bei einem Mausklick geschehen soll.
      \end{itemize}

  \subsection{Paket edu.kit.iks.Cryptographics.Vigenere}

  \subsection{Paket edu.kit.iks.Cryptographics.Vigenere.Demonstration}

  \subsection{Paket edu.kit.iks.Cryptographics.Vigenere.Experiment}

  \subsection{Paket edu.kit.iks.Cryptographics.Vigenere.Explanation}

  \subsection{Paket edu.kit.iks.Cryptographics.Caesar}
   \subsubsection{CryptoView}
          Diese Klasse besteht aus dem Programmcode, der aus den vorherigen Klassen CryptoView, das jetzt CryptoExperimentView heißt,
          und CipherDemoView, das jetzt CryptoDemonstrationView heißt, rausfaktorisiert wurde.
          
          Die beiden neuen Klassen CryptoExperimentView und CryptoDemonstrationView erben von dieser Klasse.\newline
           
    \textbf{Superklassen und Interfaces}
      \begin{itemize}
        \item  edu.kit.iks.CryptographicsLib.VisualizationView
      \end{itemize}
           
    \textbf{Methoden}
      \begin{itemize}
        \item protected CryptoView()\newline
              Konstruktor. Erzeugt die gemeinsamen Elemente von CryptoDemonstrationView und CryptoExperimentView.
        \item public void removeAlphabet()\newline
              Entfernt das Alphabet aus der Oberfläche. Public weil es oft in den Controllern
        \item protected void setupAlphabet()\newline
              Fügt das Alphabet der Oberfläche hinzu.
        \item public void removeKeyboard()\newline
              Entfernt die Tastatur für alphabetische Eingaben aus der Oberfläche.
        \item public void createKeyboard()\newline
              Fügt der Oberfläche eine Tastatur für alphabetische Eingaben hinzu.
        \item protected void setupInOutElements(char[] inputChars, int key)\newline
              Stellt die Buchstaben, die abzuarbeiten sind, voneinander getrennt in JLabels auf. 
              Auch die Textboxen, die für die Eingabe des Benutzers nötig sind werden hinzugefügt.
              Der Schlüssel wird separat neben den Buchstaben in einem JLabel dargestellt.
        \item protected void removeUserIOContainer()\newline
              Entfernt den Container für die JLabels mit den Buchstaben und die Textboxen aus der Oberlfäche.
        \item protected void removeExplanations()\newline
              Entfernt die Erklärungen aus der Oberlfäche.
	\item protected void setupExplanations(String explanations, final int flag,int yGrid, int xGrid, int widthGrid)\newline
	      Fügt der Oberfläche Erklärungen hinzu, die in dem String explanations enthalten sind.
	      Die restlichen Argumente sind werte für die Anordnung der Erklärungen in dem Layout.      
      \end{itemize}

  \subsection{Paket edu.kit.iks.Cryptographics.Caesar.Demonstration}
   Keine neuen Klassen.
  \subsection{Paket edu.kit.iks.Cryptographics.Caesar.Experiment}
   Keine neuen Klassen.
  \subsection{Paket edu.kit.iks.Cryptographics.DiffieHellman}
  
  \subsection{Paket edu.kit.iks.Cryptographics.DiffieHellman.Demonstration}

  \subsection{Paket edu.kit.iks.Cryptographics.DiffieHellman.Experiment}

% Hier kommen alle Änderungen an bestehenden Klassen rein
% (Dazu zählen auch Namensänderungen an Klassen. Schreibt dann einfach bei der Klasse hier
% rein: Klasse AlterName
% wurde zu NeuerName umbenannt
% etc...)
\section{Änderungen an Klassen}

  \subsection{Paket edu.kit.iks.Cryptographics}
    \subsubsection{HelpPopoverView}
      Keine Änderungen.

    \subsubsection{Main}
      Keine Änderungen.

    \subsubsection{MainController}
      In der Klasse MainController wurden Getter für das JFrame, sowie der StartController
      und VisualizationContainerController hinzugefügt.

    \subsubsection{StartController}
      Die Methode presentPopoverAction() erwartet nun zusätzlich den Parameter sender. Dieser gibt an, welches
      JComponent die Aktion ausgelöst hat, um den Popover korrekt darstellen zu können.

    \subsubsection{TimelinePopoverView}
      Keine Änderungen.

    \subsubsection{TimelineView}
      Keine Änderungen.

    \subsubsection{VisualizationContainerController}
      Der VisualizationContainerController unterstützt nun die Darstellung der IdlePopoverView. Hierzu wurden
      die Methoden presentIdlePopover() sowie dismissIdlePopover() hinzugefügt.\newline

    \subsubsection{VisualizationContainerView}
      Die VisualizationContainerView besitzt nun eine Content-View. Diese dient als Container für die Visualisierung
      und erlaubt es uns, den Inhalt der Visualisierung korrekt zu positionieren. Es wurde der Getter für
      beschriebene Content-View hinzugefügt.\newline

    \subsubsection{WelcomeView}
      Keine Änderungen.

  \subsection{Paket edu.kit.iks.CryptographicsLib}
    
  	\subsubsection{Klasse AbstractController}
  	
  	Die unloadView()-Methode wurde abstrakt gesetzt, um das implementieren zu erzwingen.
  	Somit kann sichergestellt werden, dass eine View auch richtig freigegeben wird. Die Methode
  	viewIsLoaded() wurde konventionsgemäß in isViewLoaded() umbenannt.

	\subsubsection{Klasse AbstractVisualizationController}
  Keine Änderungen.

	\subsubsection{Klasse AbstractVisualizationInfo}

	Die Methode getAdditionalInformationFileURL() wurde durch die Methode 
	getAdditionalInformationPath() ersetzt, die einen Pfad als String zurückgibt, und kein
	URL Objekt. Es wurde die Methode getHumanReadableDifficulty() hinzugefügt, um 
	das VisualizationDifficulty-Enum lesbar zurückzugeben.

	\subsubsection{Klasse AlphabetStripView}
	
	Es wurde ein weiterer Konstruktor hinzugefügt, mit dem man die Zellenbreite und -höhe
	bei der Initialisierung setzen kann. Die neue Methode highlight() hebt eine bestimmte
	Spalte des Alphabets an der übergebenen Stelle zur Visualisierung hervor. Die Methode 
	unHighlight() löscht die Hervorhebung wieder an der übergebenen Stelle, und unHighlightAll() 
	löscht alle Hervorhebungen.

	\subsubsection{Klasse CharacterFrequencyDiagramView}

	Der Konstruktor wurde angepasst um gleich bei der Erzeugung den zu überprüfenden Text
	entgegenzunehmen, sowie die gewünschten Ausmaße des zu erzeugenden Diagramms. Des Weiteren
	wurden Getter und Setter für die Aufkommen von Zeichen hinzugefügt.

	\subsubsection{Klasse ImageView}

	Es wurde ein Getter hinzugefügt, um das geladene Bild das dargestellt werden soll zurückzugeben.

	\subsubsection{Klasse InformationView}

	Es wurde ein Getter hinzugefügt, um den HTML Inhalt zu bekommen.

	\subsubsection{Klasse KeyboardButton}

	Diese Klasse wurde vollständig entfernt, da es durch einfache JButtons darstellbar war.

	\subsubsection{Klasse KeyboardView}

	Der Konstruktor des Keyboards wurde angepasst, um ein JTextField und einen Modus entgegenzunehmen.
	Das JTextField ist dasjenige Feld, welches durch das Keyboard manipuliert werden soll. Der zweite
	Parameter, der Modus, gibt an, ob in das Feld ein einzelner Buchstabe eingegeben werden soll, oder
	ein Text wie ein Name eingegeben werden kann. KeyboardView implementiert nun jetzt auch die
	ActionListener-Schnittstelle und implementiert ihre actionPerformed() Methode, um Tastendrücke
	des Keyboards abarbeiten und an das Textfeld weiterleiten zu können.

	\subsubsection{Klasse NumbersStripView}

	Diese Klasse wurde vollständig entfernt, da im Verlauf der Implementierung festgestellt wurde,
	dass sie nirgends benötigt wird. (Auch wurde diese Klasse teilweise durch die NumpadView ersetzt).

	\subsubsection{Klasse PopoverView}

	Die Klasse wurde um eine dismiss()-Methode ergänzt, um ein Popover wieder schließen zu können.
	Des Weiteren wurden Getter und Setter für die Umrandungsfarbe hinzugefügt.

	\subsubsection{Klasse VisualizationView}

	Es wurden lediglich Setter für den ``Next'' und den ``Back''-Button hinzugefügt.

  \subsection{Paket edu.kit.iks.Cryptographics.Vigenere}
    \subsubsection{Klasse AbstractController}

    \subsubsection{Klasse VigenereModel}

    \subsubsection{Klasse VigenereVisualizationInfo}

  \subsection{Paket edu.kit.iks.Cryptographics.Vigenere.Demonstration}
    \subsubsection{Klasse FirstDemonstrationController}

    \subsubsection{Klasse FirstDemonstrationView}

    \subsubsection{Klasse SecondDemonstrationController}

    \subsubsection{Klasse SecondDemonstrationView}

    \subsubsection{Klasse ThirdDemonstrationController}

    \subsubsection{Klasse ThirdDemonstrationView}

  \subsection{Paket edu.kit.iks.Cryptographics.Vigenere.Experiment}
    \subsubsection{Klasse FirstExperimentController}

    \subsubsection{Klasse FirstExperimentView}

    \subsubsection{Klasse SecondExperimentController}

    \subsubsection{Klasse SecondExperimentView}

  \subsection{Paket edu.kit.iks.Cryptographics.Vigenere.Explanation}
    \subsubsection{Klasse FirstExplanationController}

    \subsubsection{Klasse FirstExplanationView}

    \subsubsection{Klasse SecondExplanationController}

    \subsubsection{Klasse SecondExplanationView}

  \subsection{Paket edu.kit.iks.Cryptographics.Caesar}
  
  	\subsubsection{Klasse CaesarVisualizationInfo}
  	 Keine Änderungen.

 	\subsection{Paket edu.kit.iks.Cryptographics.Caesar.Demonstration}
          
          \subsubsection{Klasse IntroductionController}
            Die Methode animationProceed() wurde entfernt. Und durch die private Hilfsmethode proceedIntroduction() ersetzt.
            Diese erfüllt dieselbe Aufgabe und da sie zum Aufgabenbereich dieses Controllers gehört, ist ein globaler Zugriff
            unnötig.
       	  \subsubsection{Klasse IntroductionView}
            Die Methode animationStart() mitsamt ihrer privaten Hilfsfunktionen wurde in den IntroductionController ausgelagert.
            Die View dient als Container für Oberflächenelemente. Funktionalität braucht sie nicht.
	  \subsubsection{Klasse CipherDemoController}
            Diese Klasse heißt jetzt CryptoDemonstrationController.\newline
         
           \textbf{Änderungen an Methoden.}\newline
            proceedAnimation() wurde entfernt. Und durch die private Hilfsmethode demonstrate() ersetzt.
            Diese erfüllt dieselbe Aufgabe und da sie zum Aufgabenbereich dieses Controllers gehört, 
            ist ein globaler Zugriff unnötig.

	\subsubsection{Klasse CipherDemoView}
	   Diese Klasse heißt jetzt CryptoDemonstrationView.\newline
	   
	    \textbf{Änderungen an Superklassen und Interfaces.}\newline
	     Diese Klasse erbt nun von CryptoView (siehe das Kapitel ``Neue Klassen'') und nicht mehr direkt von VisualizationView.\newline
	   
	    \textbf{Änderungen an Methoden.}\newline
	     Die Methode animationStart() mitsamt ihrer privaten Hilfsfunktionen wurde in den CryptoDemonstrationController ausgelagert.

  \subsection{Paket edu.kit.iks.Cryptographics.Caesar.Experiment}

	\subsubsection{Klasse CryptoController}
          Diese Klasse heißt jetzt CryptoExperimentController.\newline
          
          \textbf{Änderungen an Methoden.}\newline
           Es sind lediglich neue Hilfsmethoden dazugekommen.
	\subsubsection{Klasse CryptoModel}
	 An dieser Klasse wurde das Entwurfsmuster Singleton angewandt. Die vielseitige Verwendung dieser Klasse 
	 ohne den Bedarf einer mehrfachen Instanziierung ermöglichte dies.\newline 
	 
	  \textbf{Änderungen an Methoden.}\newline
            \begin{itemize}
             \item getInstance() wurde hinzugefügt\newline
                   Getter für die einzige Instanz dieser Klasse. Teil des Singleton Entwurfsmusters.
             \item Die Methode checkValidChar(char) wurde aus Redundanzgründen entfernt. 
                   Dieselbe Funktionalität bietet jetzt das verschlüsseln eines Zeichens
                   durch die Methode enc und dem vergleichen mit dem zu überprüfenden char.
             \item handleInput(String) wurde aus Gründen der Namenskonventionen umbenannt 
                   in isInputValid(String). 
             \item isInputStep() wurde entfernt, weil das Model jetzt nicht mehr an einen Controller gebunden ist
                   und keine Ahnung haben soll wie weit der User in der Visualisierung vorangekommen ist.
             \item Aus den gleichen Gründen wie davor fehlt auch der setter setInputStep() weg.
             \item decryptAndCheck(String) ist aus Redundanzgründen rausgenommen worden. Die Gründe sind identisch
                   mit der Entfernung von checkValidChar(char) nur bezieht es sich jetzt auf die Verarbeitung von
                   Strings und nicht chars, wie im oberen Fall.
             \item Weil die Klasse jetzt das Entwurfsmuster Singleton verwendet, ist der Konstruktor nicht mehr public,
                   sondern private.
             \item public String enc(int key, String text) wurde hinzugefügt.\newline
                   Diese Methode verschlüsselt caesarkonform einen beliebigen Text von beliebiger Länge, mit einem beliebigen Schlüssel.
                   Der Text kann auch Sonderzeichen enthalten. Sowie auch html tags.
             \item public String dec(int key, String cipher) wurde hinzugefügt.\newline
                   Ruft enc mit einem negativen Schlüssel auf. Bentötigt bei Entschlüsselung.
             \item private int generateRandomInt(int a, int b) wurde hinzugefügt.\newline
                   Erzeugt einen ``zufälligen'' ganzzahligen Wert aus dem Intervall [a,b].
             \item public int generateKey() wurde hinzugefügt.\newline
                   Diese Methode erzeugt ``zufällig'' einen Schlüssel für Caesar Ver-/Entschlüsselung.
                   Verwendet die methode private int generateRandomInt(int a, int b) mit Grenzen [1,26].
             \item public String genRandomBlamings() wurde hinzugefügt.\newline
                   Erzeugt ``zufällig'' eine Negativmeldung zu den Interaktionen des Users
                   aus einem local Stringpool.  
             \item public String genRandomCipher(int key) wurde hinzugefügt.\newline
                   Verschlüsselt einen zufälligen Text mit dem gegebenen Schlüssel 
                   und gibt diesen aus.
             \item public String genRandomGrats() wurde hinzugefügt.\newline
                   Erzeugt ``zufällig'' eine Positivmeldung zu den Interaktionen des Users
                   aus einem local Stringpool. 
             \item public String genRandomPlainSequence() wurde hinzugefügt.\newline
                   Erzeugt ``zufällig'' ein kleines Wort aus einem local Stringpool zum verschlüsseln. 
             \item public String genRandomText() wurde hinzugefügt\newline
                   Erzeugt ``zufällig'' einen größeren Text aus einem local Stringpool zum verschlüsseln.      
            \end{itemize}

	\subsubsection{Klasse CryptoView}
	 Diese Klasse wurde in CryptoExperimentView umbenannt.\newline
	 
	 \textbf{Änderungen an Superklassen und Interfaces.}\newline
	     Diese Klasse erbt nun von CryptoView (siehe Kapitel ``Neue Klassen'') und nicht mehr direkt von VisualizationView.\newline
	   
	 \textbf{Änderungen an Methoden.}\newline
	   \begin{itemize}
	     \item public void setupExperimentCore(char[] inputChars, int key) wurde hinzugefügt.\newline
	           Diese Methode erstellt alle für den Selbstversuch notwendigen Elemente und ordnet diese im Layout an.
             \item public void createNumpad()\newline
               Fügt der Oberfläche eine Tastatur für numerische Eingaben hinzu.
             \item public void removeNumpad()\newline
               Entfernt die Tastatur für numerische Eingaben aus der Oberlfäche.
           \end{itemize}
               
	\subsubsection{Klasse HistogramController}
          \textbf{Änderungen an Methoden.}\newline
           Es sind lediglich neue Hilfsmethoden dazugekommen.
	\subsubsection{Klasse HistogramView}
	 Die Methode startAnimation() wurde entfernt, da die View als Container für Oberflächenelemente dient. 
	 Funktionalität braucht sie nicht.

  \subsection{Paket edu.kit.iks.Cryptographics.DiffieHellman}
  
    \subsubsection{Klasse AbstractController}
    Diese Klasse wurde nicht gebraucht, ergo existiert sie nicht mehr

  	\subsubsection{Klasse Model}
    Singleton-Pattern wurde rückgängig gemacht, da die unterschiedlichen Zustände
    in einer Klasse behandelt werden und somit globale Variablen (also Singletons)
    unnötig werden

    \textbf{Änderungen an Methoden.}\newline
	   \begin{itemize}
           \item Konstruktor ist nicht mehr privat\newline
           \item public Model getInstance() wurde entfernt\newline
           \item public Color getPublicColor() wurde hinzugefügt\newline
           \item public void setPublicColor() wurde hinzugefügt\newline
           \item public void getAlicePublicColor() wurde entfernt\newline
           \item public void setAlicePublicColor() wurde entfernt\newline
           \item public Color getSharedColor() wurde entfernt\newline
           \item public void setSharedColor(Color setSharedColor) wurde entfernt\newline
           \item public void getBobPublicColor() wurde entfernt\newline
           \item public Color setBobPublicColor(Color bobPublicColor) wurde entfernt\newline
           \end{itemize}

	\subsubsection{Klasse DHVisualizationInfo}
    Hat sich entwurfstechnisch nicht geändert

  \subsection{Paket edu.kit.iks.Cryptographics.DiffieHellman.Demonstration}

	\subsubsection{Klasse ExplainAimController}
	    \textbf{Änderungen an Superklassen und Interfaces.}\newline
	   \begin{itemize}
            \item Diese Klasse erbt jetzt von AbstractVisualizationController statt AbstractController\newline
           \end{itemize}

	\subsubsection{Klasse ExplainAimView}
	    \textbf{Änderungen an Superklassen und Interfaces.}\newline
	   \begin{itemize}
            \item Diese Klasse erbt jetzt von VisualizationView statt JPanel\newline
           \end{itemize}
    \textbf{Änderungen an Methoden.}\newline
	   \begin{itemize}
           \item public ColorChannel getColorChannel() wurde hinzugefügt\newline
           \item public String getHelp() wurde hinzugefügt\newline
           \end{itemize}

	\subsubsection{Klasse OnewayController}
	    \textbf{Änderungen an Superklassen und Interfaces.}\newline
	   \begin{itemize}
            \item Diese Klasse erbt jetzt von AbstractVisualizationController statt AbstractController\newline
           \end{itemize}

	\subsubsection{Klasse OnewayView}
	    \textbf{Änderungen an Superklassen und Interfaces.}\newline
	   \begin{itemize}
            \item Diese Klasse erbt jetzt von VisualizationView statt JPanel\newline
           \end{itemize}
    \textbf{Änderungen an Methoden.}\newline
	   \begin{itemize}
           \item public String getHelp() wurde hinzugefügt\newline
           \end{itemize}

	\subsubsection{Klasse ExplainKeyExchangeController}
    Diese Klasse wurde entfernt

	\subsubsection{Klasse ExplainKeyExchangeView}
    Diese Klasse wurde entfernt

	\subsubsection{Klasse AliceChooseSecretController}
    Diese Klasse wurde in DHDemoController umbenannt
	    \textbf{Änderungen an Superklassen und Interfaces.}\newline
	   \begin{itemize}
            \item Diese Klasse erbt jetzt von AbstractVisualizationController statt AbstractController\newline
           \end{itemize}

	\subsubsection{Klasse AliceChooseSecretView}
    Diese Klasse wurde in DHDemoView umbenannt

	\subsubsection{Klasse BobChooseSecretController}
    Diese Klasse wurde entfernt

	\subsubsection{Klasse BobChooseSecretView}
    Diese Klasse wurde entfernt

	\subsubsection{Klasse MixColorController}
    Diese Klasse wurde entfernt

	\subsubsection{Klasse MixColorView}
    Diese Klasse wurde entfernt

	\subsubsection{Klasse DemoOneWayController}
    Diese Klasse wurde neu hinzugefügt
    Der Controller der DemoOneWayView
	    \textbf{Änderungen an Superklassen und Interfaces.}\newline
	   \begin{itemize}
            \item Diese Klasse erbt jetzt von AbstractVisualizationController statt AbstractController\newline
           \end{itemize}

	\subsubsection{Klasse DemoOneWayView}
    Diese Klasse wurde neu hinzugefügt
    Diese View zeigt wie Einwegfunktionen agieren
	    \textbf{Änderungen an Superklassen und Interfaces.}\newline
	   \begin{itemize}
            \item Diese Klasse erbt von VisualizationView statt JPanel
           \end{itemize}
    \textbf{Änderungen an Methoden.}\newline
	   \begin{itemize}
           \item public DemoOneWayView() wurde hinzugefügt\newline
               Der Konstruktor positioniert seine Subkomponenten etc.
           \item public String getHelp() wurde hinzugefügt\newline
               Gibt einen Hilfe-Zeichenkette zurück
           \end{itemize}


	\subsubsection{Klasse ColorMix}
    Diese Klasse wurde neu hinzugefügt
    Ihre Aufgabe ist es als JPanel grafisch Farben zu mischen
	    \textbf{Änderungen an Superklassen und Interfaces.}\newline
	   \begin{itemize}
            \item Diese Klasse erbt von JPanel\newline
           \end{itemize}
    \textbf{Änderungen an Methoden.}\newline
	   \begin{itemize}
           \item public ColorMix(int circleSize, Dimension dimension) wurde hinzugefügt\newline
           \item public void mixColors(boolean mix, boolean repeat, final NextStepCallback cb) wurde hinzugefügt\newline
               Mischt grafisch zwei Farben zusammen
           \item public void seperateColors(boolean seperate, boolean repeat) wurde hinzugefügt\newline
               Trennt grafisch zwei Farben
           \item public void setEllipColor(int which, Color color) wurde hinzugefügt\newline
               Setzt erste oder zweite Farbe auf color
           \item public Color getMixedColor() wurde hinzugefügt\newline
               Gibt die gemischte Farbe zurück
           \item public boolean isComputeFinalMix() wurde hinzugefügt\newline
               Falls wahr, wird eine andere Formel zum Farbenmischen verwendet
           \item public void setComputeFinalMix(boolean computeFinalMix)\newline
               Setzt den Wert von computeFinalMix
           \end{itemize}

	\subsubsection{Klasse ColorChannel}
    Diese Klasse wurde neu hinzugefügt
    Ihre Aufgabe ist es als JPanel grafisch den
    Kommunikationskanal zwischen Alice, Bob und Eve
    zu vermitteln und Farben über diesen Kanal zu visuell
    zu verschicken.
	    \textbf{Änderungen an Superklassen und Interfaces.}\newline
	   \begin{itemize}
            \item Diese Klasse erbt von JPanel\newline
           \end{itemize}
    \textbf{Änderungen an Methoden.}\newline
	   \begin{itemize}
           \item public ColorChannel(Dimension d, int circleSize) wurde hinzugefügt\newline
               Der Konstruktor nimmt die Dimension und die Größe der Kreise als Parameter,
               und errechnet daraus wie die restlichen Elemente positioniert werden
           \item public void sendToBob(final NextStepCallback cb, final boolean keepFirst) wurde hinzugefügt\newline
               Schickt eine Farbkreis von Alice zu Bob, cb ist der Callback der aufgerufen wird, wenn dies
               erledigt wurde.
               Wenn keepFirst wahr ist, wird die Farbe neben Alice dauerhaft dargestellt
           \item public void sendToAlice(final NextStepCallback cb, final boolean keepFirst) wurde hinzugefügt\newline
               Tut das gleiche wie sendToBob, nur dass von Bob nach Alice die Farbe verschickt wird und
               wenn keepFirst den Wert wahr hat, wird die Farbe neben Bob visualisiert
           \item public void chooseColorToKeep(Color color, int who) wurde hinzugefügt\newline
               Ein Farbkreis mit der Farbe color wird neben who hinzugefügt also neben alice, bob oder eve
           \item public void stopTimer() wurde hinzugefügt\newline
               Stoppt den Timer. Muss aufgerufen werden wenn ColorChannel nicht mehr verwendet wird,
               da sonst der Timer weiterlaufen könnte
           \item public void loadView() wurde hinzugefügt\newline
               Tut im Grunde das gleiche wie im Konstruktor, nur dass diese Funktion dafür
               aufgerufen werden muss, wenn sich die Größe ändert
           \item public void choosePublicColor(Color color) wurde hinzugefügt\newline
               Setzt die öffentliche Farbe
           \item public void chooseAlicePrivateColor(Color color) wurde hinzugefügt\newline
               Setzt die private Farbe von Alice
           \item public void chooseBobPrivateColor(Color color) wurde hinzugefügt\newline
               Setzt die private Farbe von Bob
           \item public void mixAlicePrivatePublic() wurde hinzugefügt\newline
               Mischt die private Farbe von Alice und die öffentliche Farbe
               und speichert diese
           \item public void mixBobPrivatePublic() wurde hinzugefügt\newline
               Mischt die private Farbe von Bob und die öffentliche Farbe
               und speichert diese
           \item public void sendPublicColor() wurde hinzugefügt\newline
               Sendet die öffentliche Farbe von Alice zu Bob
           \item public void sendAliceMixedColorToBob(NextStepCallback cb) wurde hinzugefügt\newline
               Sendet die Mixtur von Alice zu Bob, und ruft cb auf wenn dies geschehen ist
           \item public void sendBobMixedColorToAlice(NextStepCallback cb) wurde hinzugefügt\newline
               Sendet die Mixtur von Bob zu Alice , und ruft cb auf wenn dies geschehen ist
           \item public void mixAliceFinalSecret(NextStepCallback cb) wurde hinzugefügt\newline
               Mischt das gemeinsame Geheimnis zusammen und ruft cb auf, wenn dies geschehen ist
           \item public void mixBobFinalSecret(NextStepCallback cb) wurde hinzugefügt\newline
               Mischt das gemeinsame Geheimnis zusammen und ruft cb auf, wenn dies geschehen ist
           \item public boolean isRepeat() wurde hinzugefügt\newline
               Wahr wenn das Senden in einer periodischen Animation laufen soll
           \item public void setRepeat(boolean repeat) wurde hinzugefügt\newline
               Setzt den repeat Wert
           \item public Color getColor() wurde hinzugefügt\newline
               Gibt die Farbe die als nächstes versendet wird zurück
           \item public void setColorNextToSend(Color color) wurde hinzugefügt\newline
               Setze die Farbe die als nächstes versendet werden soll
           \item public boolean isKeepColor() wurde hinzugefügt\newline
               Wenn Wahr, werden die verschickten Farben neben den Empfängern visualisiert
           \item public Color getPublicColor() wurde hinzugefügt\newline
               Gibt die öffentliche Farbe zurück
           \item public Color getAlicePrivateColor() wurde hinzugefügt\newline
               Gibt die private Farbe von Alice zurück
           \item public Color getAliceMixedColor() wurde hinzugefügt\newline
               Gibt die Mixtur von öffentlicher und privater Farbe von Alice zurück
           \item public Color getBobPrivateColor() wurde hinzugefügt\newline
               Gibt die private Farbe von Bob zurück
           \item public Color getBobMixedColor() wurde hinzugefügt\newline
               Gibt die Mixtur von öffentlicher und privater Farbe von Bob zurück
           \end{itemize}

	\subsubsection{Klasse ColorChooser}
    Diese Klasse wurde neu hinzugefügt
    Dieses JPanel erlaubt uns eine Farbe auszuwählen
	    \textbf{Änderungen an Superklassen und Interfaces.}\newline
	   \begin{itemize}
            \item Diese Klasse erbt von JPanel\newline
           \end{itemize}
    \textbf{Änderungen an Methoden.}\newline
	   \begin{itemize}
           \item public ColorChooser(Dimension d, Color color, Color[] colors) wurde hinzugefügt\newline
               Der Konstruktor nimmt die Größe des JPanel als Parameter,
               die Farbe die als erstes angezeigt werden soll, nämlich color,
               und eine Liste an Farben aus denen man seine Farbe wählen kann colors
           \item public Color getCurrentColor wurde hinzugefügt\newline
               Gibt die momentan ausgewählte Farbe zurück
           \item public void setToChooseFrom(Color[] colors) wurde hinzugefügt\newline
               Setze die Farben die man wählen kann neu
           \item public Color[] getToChooseFrom() wurde hinzugefügt\newline
               Gebe die Farben zurück, die man wählen kann
           \end{itemize}

	\subsubsection{Klasse Ellipse2DwithColor}
    Diese Klasse wurde neu hinzugefügt
    Diese Klasse ist ein Wrapper um Ellipse2D um
    sie zusätzlich mit einer Farbe zu assozieren
	    \textbf{Änderungen an Superklassen und Interfaces.}\newline
	   \begin{itemize}
            \item Diese Klasse erbt von Ellipse2D\newline
           \end{itemize}
    \textbf{Änderungen an Methoden.}\newline
	   \begin{itemize}
           \item public Ellipse2DwithColor(double x, double y, double w, double h) wurde hinzugefügt\newline
               Gleicht dem Konstruktor der Oberklasse und setzt die Farbe auf Schwarz
           \item public Ellipse2DwithColor(double x, double y, double w, double h, Color color) wurde hinzugefügt\newline
               Gleicht dem Konstruktor der Oberklasse und setzt die Farbe auf color
           \item public Color getColor() wurde hinzugefügt\newline
               Gibt die aktuelle Farbe zurück
           \item public void setColor(Color color) wurde hinzugefügt\newline
               Setzt einen neuen Wert für die aktuelle Farbe
           \end{itemize}

	\subsubsection{Interface NextStepCallback}
    Diese Interface wurde neu hinzugefügt
    Dient als Imitator von Funktionen höherer Ordnung,
    da Java diese nicht hat
    \textbf{Änderungen an Methoden.}\newline
	   \begin{itemize}
           \item public void callback() wurde hinzugefügt\newline
               Unser Callback
           \end{itemize}

  \subsection{Paket edu.kit.iks.Cryptographics.DiffieHellman.Experiment}

	\subsubsection{Klasse YourTurnController}
	    \textbf{Änderungen an Superklassen und Interfaces.}\newline
        \begin{itemize}
            \item Diese Klasse erbt jetzt von AbstractVisualizationController statt AbstractController\newline
        \end{itemize}

	\subsubsection{Klasse YourTurnView}
	    \textbf{Änderungen an Superklassen und Interfaces.}\newline
        \begin{itemize}
            \item Diese Klasse erbt jetzt von VisualizationView statt JPanel\newline
        \end{itemize}

	\subsubsection{Klasse DHExperimentController}
    Diese Klasse wurde neu hinzugefügt
    Sie ist der zur DHExperimentView korrespondierende Controller
	    \textbf{Änderungen an Superklassen und Interfaces.}\newline
        \begin{itemize}
            \item Diese Klasse erbt jetzt von AbstractVisualizationController statt AbstractController\newline
        \end{itemize}

	\subsubsection{Klasse DHExperimentView}
    Diese Klasse wurde neu hinzugefügt
    Dient als View zum DH Experiment
	    \textbf{Änderungen an Superklassen und Interfaces.}\newline
        \begin{itemize}
            \item Diese Klasse erbt von JPanel\newline
        \end{itemize}
    \textbf{Änderungen an Methoden.}\newline
	   \begin{itemize}
           \item public DHExperimentView() wurde hinzugefügt\newline
               Der Konstruktor erstellt seine Subkomponenten und
               positioniert diese Korrekt
           \item public void setRemember(ActionListener remember) wurde hinzugefügt\newline
               Merkt sich einen ActionListener damit die View seinem Button den ursprünglichen
               ActionListener wieder setzen kann
           \item public String getHelp() wurde hinzugefügt\newline
               Gibt einen Hilfe-Zeichenkette zurück
           \end{itemize}

	\subsubsection{Klasse CongratsController}
    Diese Klasse wurde neu hinzugefügt
    Sie ist der zur CongratsView korrespondierende Controller
	    \textbf{Änderungen an Superklassen und Interfaces.}\newline
        \begin{itemize}
            \item Diese Klasse erbt von AbstractVisualizationController \newline
           \end{itemize}

	\subsubsection{Klasse CongratsView}
    Diese Klasse wurde neu hinzugefügt
	    \textbf{Änderungen an Superklassen und Interfaces.}\newline
        \begin{itemize}
            \item Diese Klasse erbt von JPanel \newline
           \end{itemize}
    \textbf{Änderungen an Methoden.}\newline
	   \begin{itemize}
           \item public CongratsView() wurde hinzugefügt
               Der Konstruktor erstellt und positioniert seine Elemente
           \item public String getHelp() wurde hinzugefügt
               Gebe eine Hilfe-Zeichenkette zurück
           \end{itemize}

	\subsubsection{Klasse ChoosePublicColorController}
    Diese Klasse wurde entfernt

	\subsubsection{Klasse ChoosePublicColorView}
    Diese Klasse wurde entfernt

	\subsubsection{Klasse SendColorController}
    Diese Klasse wurde entfernt

	\subsubsection{Klasse SendColorView}
    Diese Klasse wurde entfernt

	\subsubsection{Klasse ChooseSecretColorController}
    Diese Klasse wurde entfernt

	\subsubsection{Klasse ChooseSecretColorView}
    Diese Klasse wurde entfernt

	\subsubsection{Klasse SendRightColorController}
    Diese Klasse wurde entfernt

	\subsubsection{Klasse SendRightColorView}
    Diese Klasse wurde entfernt

	\subsubsection{Klasse MixFinalSecretController}
    Diese Klasse wurde entfernt

	\subsubsection{Klasse MixFinalSecretView}
    Diese Klasse wurde entfernt
	
 \restoregeometry

\glsaddall
\printglossary[numberedsection, style=altlist]

\end{document}
