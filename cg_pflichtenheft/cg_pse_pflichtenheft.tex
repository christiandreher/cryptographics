\documentclass{article}

\usepackage[utf8]{inputenc}
\usepackage[ngerman]{babel}
\usepackage[T1]{fontenc}
\usepackage{enumitem}

\newlist{FA}{enumerate}{1}
\setlist[FA]{label=/FA\arabic*/}

\newlist{NA}{enumerate}{1}
\setlist[NA]{label=/NA\arabic*/}


\title{Cryptographics Pflichtenheft}
\author{}
\date{\today}

\begin{document}

\maketitle
\tableofcontents
\newpage

\section{Zielbestimmung}

Im Rahmen der PSE-Veranstaltung soll für das Kryptologikum des Instituts für 
Kryptographie und Sicherheit eine Software (Cryptographics / Anicrypto?) zur 
Demonstration kryptographischer Verfahren erstellt werden. \\
\\
Das Programm soll im Laufe der Ausstellung Kryptologikum das 
Interesse der Besucher wecken, 
sich mit Verschlüsselung zu befassen und 
schließlich auch ausgewählte Inhalte der Kryptographie näher bringen
\\
\\
Soll Spaß machen.

\subsection{Musskriterien}

\begin{itemize}
    \item Implementierung der Caesar-Chiffre
    \item Implementierung Vigen`ere-Chiffre
    \item Implementierung RSA
    \item Implementierung Diffie-Hellmann
    \item Implementierung Hashfunktionen
    \item Robuste Programmierung
    \item Intuitive Benutzerführung
    \item Schnelle Reaktionszeiten
    \item Zugriff auf Krypto-Verfahren über Zeitleiste:
        \begin{itemize}
            \item Pop-Up mit kurzem Umriss des Verfahrens
            \item Draufklicken um zum Verfahren zu Gelangen
            \item Einfärbung grün/gelb/rot nach Komplexität des Verfahrens
        \end{itemize}
    \item Drei Optionen je Verfahren: Demo, Selsbstversuch, weitere Informationen
    \item Angeleiter Selbstversuch bei "roten" Verfahren
    \item Interaktion des Benutzers über den Touchscreen
    \item einfache, übersichtliche UI, die sich gut auf Touchscreen bedienen lässt
    \item Robustheit gegen au"sergewöhnliche Interaktion
    \item Man soll das Programm nicht schlie"sen können währen der Vorführung
    \item Praxisbezug bei modernen Kryptoverfahren
    \item Sammlung wiederverwendbare UI-Element um z.B. die Texteingabe gleich
        zu behandeln
    \item Krypto-Algorithmen lassen sich schrittweise ausführen, um den Prozess
        langsam zu verdeutlichen (bspw. bei Caesar: Buchstabe für Buchstabe
        bis Codewort generiert wurde)
    \item der Benutzer soll die Möglichkeit bekommen Dinge Schritt für Schritt
        in seiner eigenen Geschwindigkeit nachzuvollziehen und ggf.
        zurückspringen können

    \item Es darf kein Zurückkommen zum Betriebssystem möglich sein
    \item Visualisierungen sollten Grafiken und Animationen verwenden um Vorgänge zu verdeutlichen
\end{itemize}

\subsection{Wunschkriterien}

\begin{itemize}
    \item Implementierung Public Key – Infrastruktur
    \item Implementierung Shamir Secret Sharing
    \item Implementierung Passwortsicherheit
    \item Implementierung One-Time-Pad
    \item Implementierung Blockchiffre (DES, AES)
    \item Bei Problemen im Selbstversuch hilft der Computer mit Tipps aus
    \item Modulares Austauschen/Hinzufügen von weiteren kryptografischen Verfahren
    \item Einschätzung des aktuellen Wissenstands des Benutzers.
    \item Implementierung elliptischer Kurven
    \item Literaturhinweise(auch weiterführende Literatur), sofern Interesse besteht.
    \item Visualisierungen sollten möglichst ansprechend sein, indem sie z.B. Analogien 
        verwenden (ein Datenpaket wird zu einer Postkarte, ein Schlüssel wird zum einem tatsächlichen Schlüssel, ...).
    \item Wenn über einen Zeitraum x keine Nutzereingaben erfolgen erscheint ein Countdown 
        nach dessen Ablauf das Programm auf den Willkommensbildschirm zurücksetzt
    \item Visualisierung von Modulo mit einer Uhr
    \item Visualisierung von Angriffen auf Verfahren
    \item (Blockchiffre im ECB-Modus vs. CBC-Modus, siehe Bild in 
    \item http://de.wikipedia.org/wiki/Electronic\_Code\_Book\_Mode)
    \item Brute-Force Angriff auf (symmetrisch/asymmetrisch?) Verschlüsselung
    \item Enigma?
    \item Zahlentheorie: modulares Rechnen/ Faktorisierungsproblem/ RSA
\end{itemize}

\subsection{Abgrenzungskriterien}
\begin{itemize}
    \item Sämtliche kryptologischen Verfahren werden nur zu Vorführungszwecken implementiert. Eine sichere Implementierung ist nicht vorgesehen.
    \item Dies ist ein Vorführprogramm für eine Ausstellung, demnach ist eine andere  Verwendung nicht ratsam, 
        da dafür ganz andere Anforderungen gelten, die hier nicht beachtet werden. (umformulieren!)
    \item keine optimierte und 100\% standardkonforme Implementierung von Krypto-Algorithmen, 
        sondern Fokus auf schrittweiser Ausführbarkeit
    \item keine formale Korrektheit bei Erklärungen sondern Ansatz ohne nötige Vorkenntnisse 
        um breiter Masse zugänglich zu sein
\end{itemize}

\section{Produkteinsatz}
\subsection{Anwendungsbereiche}
Cryptographics soll in erster Linie als Ausstellungsstück für das Kryptologikum des Instituts für Kryptographie und Sicherheit (IKS) 
am Karlsruher Institut für Technologie (KIT) dienen. 

Besuchern der Ausstellung soll das Funktionsprinzip und die Verwendung historischer, 
sowie aktueller, kryptographischer Verfahren näher   gebracht werden. 
Diese sollen anhand von vereinfachten und beispielhaften Szenarien aus dem Alltag vermittelt werden, 
mit dem Ziel ein größeres Interesse an der Materie zu wecken.

Cryptograhics soll primär auf dem eeePC im Kryptologikum eingesetzt werden. Portabilität wäre jedoch wünschenswert.
\\ 
\\
- Ausstellung.

\subsection{Zielgruppen}
Das Programm richtet sich insbesondere an Kinder, Jugendliche und Erwachsene mit grundlegenden Kenntnissen der Mathematik. 
Die Zielgruppe ist demnach ein anonymes Publikum, das sich potenziell aus fachfremden Personen zusammensetzt. 
Deshalb dürfen für die Verwendung von Cryptographics keine Vorkenntnisse in der Kryptographie vorausgesetzt werden.
\\
\\
- Besucher einer kryptographischen Ausstellung mit unterschiedlichem Wissensstand und Interesse.

\subsection{Betriebsbedingungen}

Das Tool soll im Dauerbetrieb problemlos laufen. 
Es dürfen keine Ausnahmen auftreten, die den Betrieb des Tools behindern, 
oder gar abstürzen lassen und auf den Desktop zurückzukehren. 
Der Betrieb soll für den eeePC optimiert sein.
\\
\\
Optimale Ausführung auf gegebener Hardware (Intel Atom, Touchscreen)
\\
\\
-Einsatz im Rahmen einer Ausstellung.
\\
Kein Abstürzen bei unkontrollierten, „wilden“ Nutzereingaben
\\
System darf nicht manipuliert werden, d.h. keine oder nur geschützte
Möglichkeit um Programm zu beenden
\\
\\
reine Offline-Anwendung. Darf keine Internetanbindung benötigen
\section{Produktumgebung}

Software/Hardware… => Weiß ich grad nicht. XP auf nem eeePc
Einsatz auf eeePC wieauchimmer (keine Ahnung wie die Kiste heißt)
- Produktumgebung ist ein Windows-PC mit reisistivem Touchscreen 
und Windows XP

\subsection{Software}

\subsection{Hardware}

\section{Produktfunktionen}

\subsection{Funktionale Anforderungen}
\begin{FA}[start=100]
\item Kryptographische Verfahren auswählen
\end{FA}

\begin{FA}[start=200]
\item Funktionen von den jeweiligen Verfahren abhängig
\end{FA}

\begin{FA}[start=300]
\item Jederzeit die Möglichkeit für interaktive Hilfe ([?]-Button)
\end{FA}

- Funktionen: Willkommensbildschirm, Algo/Visualisierung wählen, Algo bearbeiten, 
Soforthilfe passend zum Algo falls man nicht weiter weiß, Algo beenden, 
ggf. Kontrollfragen oder selbstständiges Anwenden des Gelernten am Ende (z.B. ein Wort mit Caesar verschlüsseln)
\\
\\
RSA und Public-Key Verschlüsselung anhand von Pad-Locks als 
öffentliche Schlüssel und den Schlüssel  dazu als privaten,
vielleicht mit Komplementärfarben visualisieren ähnlich wie Diffie-Hellman

\subsection{Nichtfunktionale Anforderungen}

\begin{NA}[start=100]
\item Schnelle Reaktionszeit
\end{NA}
\begin{NA}[start=200]
\item Große Robustheit
\end{NA}
\begin{NA}[start=300]
\item Intuitive Benutzerführung
\end{NA}

- Die Anwendung muss möglichst performant und verzögerungsfrei auf Benutzereingaben reagieren (soweit das die HW zulässt...)
\\
\\
- Die Anwendung muss leicht zu bedienen sein

\section{Produktdaten}

\subsection{Nutzerdaten}

- zu speichern sind ggf. Nutzungsstatistiken, um herauszufinden, welche Visualisierung gut ankommt und welche nicht. 
Dies erlaubt das Ausstellungsstück zielgerichtet weiterzuentwickeln.

\section{Benutzerschnittstelle}
[Grafiken, Entwürfe]

\section{Globale Testfälle}
- automatisiertes Testen mithilfe von JUnit vor allem der Krypto-Algorithmen
- ggf. automatisiertes Testen der UI mithilfe von Szenarien und geeignetem Framework
Stresstest mit möglichst zufälligen und willkürlichen Eingaben (wie es Kinder eben tun)
Usability Tests mit passenden Testpersonen

\section{Beispielszenarien}
\begin{itemize}
    \item Beispiel läuft schrittweise ab. Problem wird dargestellt: A möchte B eine Nachricht schicken, die nicht von einer anderen Person als B gelesen werden soll. C versucht die Nachricht zu bekommen. Dabei können A und B nur über ein unsicheres Medium kommunizieren (Analogie z.B. schlüssel per Post verschicken), das von C abgehört wird.
itemA und B generieren ihre Schlüsselpaare. Der genaue Schlüssel ist nicht relevant.
\item A fragt B nach seinem public key, den B auch bereitwillig mitteilt. C hört den public key von B.
\item A verschlüsselt die Nachricht mit dem public key von B und verschickt die Nachricht an B. C hört die verschlüsselte Nachricht ab.
\item B entschlüsselt die Nachricht von A mit seinem private key und kann die Nachricht lesen
\item C versucht mit der Nachricht irgendetwas anzufangen und versucht sie mit dem abgehörten public key von B zu entschlüsseln. Das scheitert natürlich.

Beispiel einer Visualisierung anhand des Caesar-Algos:

\item Es wird erklärt, dass vermutlich Caesar dieses Verfahren verwendet hat um geheime Nachrichten zu verschlüsseln
\item Erklärung des Prinzips an einem Beispielsatz. Es werden schrittweise alle gleichen Buchstaben markiert und in ihr Äquivalent umgewandelt (z.B. "Hallo, wie geht's?" -> zuerst alle H zu K, dann alle a zu d, dann alle l zu o). Hier gibt es die Möglichkeit, schrittweise durchzugehen und noch mal zurück zu gehen. Außerdem kann man wenn man das Prinzip verstanden hat ans Ende springen.
\item Als nächstes muss die Person ein Wort selbst verschlüsseln mit Vorschrift (z.B. a -> b).
\item Am Ende wird mithilfe eines Diagrammes und der Kenntnis, das E der häufigste Buchstabe im Deutschen ist gezeigt, dass man diese Verschlüsselung leicht umgehen kann
\end{itemize}
\section{Qualitätsbestimmung}

\end{document}


