\documentclass{article}

\usepackage[utf8]{inputenc}
\usepackage[ngerman]{babel}
\usepackage[ngerman]{translator}
\usepackage[T1]{fontenc}
\usepackage{enumitem}
\usepackage{graphicx}
\usepackage{geometry}
\usepackage{float}
\usepackage{url}
\usepackage[bottom]{footmisc}
\usepackage{hyperref}
\usepackage[nonumberlist, section=subsection]{glossaries}
\usepackage{tabularx}

\usepackage{pifont}
\newcommand{\cmark}{\ding{51}}

\title{\textbf{Validierungsphase} \\ Cryptographics}
\author{}
\date{\today}

%Glossar-Befehle anschalten
\makeglossaries
% \newglossaryentry{identifier}{name={Name}, description={Description}}

\begin{document}

% The cover page.
\maketitle
\begin{table}[b]
  \begin{tabular}{| l | l | l |}
    \hline
    \textbf{Phase} & \textbf{Verantwortlicher} & \textbf{Email} \\ \hline
    Pflichtenheft & Matthias Jaenicke & matthias.jaenicke@student.kit.edu \\ \hline
    Entwurf & Matthias Plappert & undkc@student.kit.edu \\
            & Julien Duman & uncyc@student.kit.edu \\ \hline
    Implementierung & Christian Dreher & uaeef@student.kit.edu \\ \hline
    Qualitätssicherung & Wasilij Beskorovajnov & uajkm@student.kit.edu \\ \hline
    Präsentation & Aydin Tekin & aydin.tekin@student.kit.edu \\ \hline
    \end{tabular}
\end{table}
\thispagestyle{empty}
\newpage

% Table of contents page.
\tableofcontents
\newpage

% Start of the actual document.
\section{Einleitung}
text here

\section{Testszenarien}
text here
  \subsection{Szenario 1}
    \begin{table}[H]
      \begin{tabularx}{\textwidth}{| >{\raggedright\arraybackslash}X | c |}
        \hline
        \textbf{Funktion} & \textbf{Erfüllt?} \\
        \hline
        Startbildschirm wird angezeigt & \cmark \\
        \hline
        Zeitleiste wird angezeigt & \cmark \\
        \hline
        Verfahren werden auf Zeitleiste dargestellt & \cmark \\
        \hline
        Verfahren sind farblich gekennzeichnet & \cmark \\
        \hline
        Caesar-Verfahren kann ausgewählt werden & \cmark \\
        \hline
        Popover mit Zusammenfassung wird dargestellt & \cmark \\
        \hline
        Verfahren kann gestartet werden & \cmark \\
        \hline
        Verfahren erlaubt mehrschrittige Navigation & \cmark \\
        \hline
        Verfahren enthält Einleitung & \cmark \\
        \hline
        Selbstversuch zur Verschlüsselung funktioniert & \cmark \\
        \hline
        Selbstversuch zur Entschlüsselung funktioniert & \cmark \\
        \hline
        Rückkehr zum Startbildschirm ist möglich & \cmark \\
        \hline
        Auswahl eines anderen Verfahrens ist möglich & \cmark \\
        \hline
        Am Ende des Verfahrens werden weitere Informationen sowie QR-Code dargestellt & \cmark \\
        \hline
        QR-Code enthält Link zu weiteren Informationen & \cmark \\
        \hline
      \end{tabularx}
    \end{table}

  \subsection{Szenario 2}
    \begin{table}[H]
      \begin{tabularx}{\textwidth}{| >{\raggedright\arraybackslash}X | c |}
        \hline
        \textbf{Funktion} & \textbf{Erfüllt?} \\
        \hline
        Zufällige Eingabe bringt das Programm nicht zum Abstürzen & \cmark \\
        \hline
        Verfahren kann ausgewählt werden & \cmark \\
        \hline
        Programm hat jederzeit einen Button um zum Startbildschirm zurückzukehren & \cmark \\
        \hline
      \end{tabularx}
    \end{table}

  \subsection{Szenario 3}
    \textit{Hinweis: Die Variablen x und y richten sich nach der aktuellen Konfiguration des Programmes. Hierbei
      ist x der Wert des Parameters \texttt{idleTimeout} sowie y der Wert des Parameters \texttt{resetTimeout}.}
    \newline

    \begin{table}[H]
      \begin{tabularx}{\textwidth}{| >{\raggedright\arraybackslash}X | c |}
        \hline
        \textbf{Funktion} & \textbf{Erfüllt?} \\
        \hline
        Verfahren kann ausgewählt werden & \cmark \\
        \hline
        Programm erkennt fehlende Eingabe nach x Sekunden & \cmark \\
        \hline
        Programm zeigt Warnhinweis, dass sich das Programm in y Sekunden zurücksetzen wird, dar & \cmark \\
        \hline
        Nach y Sekunden kehrt das Programm automatisch zum Startbildschirm zurück & \cmark \\
        \hline
      \end{tabularx}
    \end{table}

  \subsection{Szenario 4}

    \begin{table}[H]
      \begin{tabularx}{\textwidth}{| >{\raggedright\arraybackslash}X | c |}
        \hline
        \textbf{Funktion} & \textbf{Erfüllt?} \\
        \hline
        Programm läuft im Vollbildmodus & \cmark \\
        \hline
        Programm hat keine Möglichkeit um per Touchscreen-Eingabe zum Betriebssystem zurückzukehren & \cmark \\
        \hline
        Programm kann mithilfe einer angeschlossenen Hardwaretastatur beendet werden & \cmark \\
        \hline
      \end{tabularx}
    \end{table}

\section{Coverage-Tests}
text here
  % They devide it like this on their table of contents,
  % but since we only have juint test cases, this might be 
  % senseless?
  \subsection{Abdeckung der JUnit-Testfälle}
    \subsubsection{Klasse ABC}
    \subsubsection{Klasse XYZ}
  \subsection{Abdeckung aller Testfälle}

\section{Regressionstests}
text here

\section{Beschreibung der Fehler}
text here

% We may need this, or not...
\section{Anhang}
text here

\section{Glossar}
text here

 \restoregeometry

\glsaddall
\printglossary[numberedsection, style=altlist]

\end{document}
